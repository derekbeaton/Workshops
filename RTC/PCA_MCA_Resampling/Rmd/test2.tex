\PassOptionsToPackage{unicode=true}{hyperref} % options for packages loaded elsewhere
\PassOptionsToPackage{hyphens}{url}
%
\documentclass[
  ignorenonframetext,
]{beamer}
\usepackage{pgfpages}
\setbeamertemplate{caption}[numbered]
\setbeamertemplate{caption label separator}{: }
\setbeamercolor{caption name}{fg=normal text.fg}
\beamertemplatenavigationsymbolsempty
% Prevent slide breaks in the middle of a paragraph:
\widowpenalties 1 10000
\raggedbottom
\setbeamertemplate{part page}{
  \centering
  \begin{beamercolorbox}[sep=16pt,center]{part title}
    \usebeamerfont{part title}\insertpart\par
  \end{beamercolorbox}
}
\setbeamertemplate{section page}{
  \centering
  \begin{beamercolorbox}[sep=12pt,center]{part title}
    \usebeamerfont{section title}\insertsection\par
  \end{beamercolorbox}
}
\setbeamertemplate{subsection page}{
  \centering
  \begin{beamercolorbox}[sep=8pt,center]{part title}
    \usebeamerfont{subsection title}\insertsubsection\par
  \end{beamercolorbox}
}
\AtBeginPart{
  \frame{\partpage}
}
\AtBeginSection{
  \ifbibliography
  \else
    \frame{\sectionpage}
  \fi
}
\AtBeginSubsection{
  \frame{\subsectionpage}
}
\usepackage{lmodern}
\usepackage{amssymb,amsmath}
\usepackage{ifxetex,ifluatex}
\ifnum 0\ifxetex 1\fi\ifluatex 1\fi=0 % if pdftex
  \usepackage[T1]{fontenc}
  \usepackage[utf8]{inputenc}
  \usepackage{textcomp} % provides euro and other symbols
\else % if luatex or xelatex
  \usepackage{unicode-math}
  \defaultfontfeatures{Scale=MatchLowercase}
  \defaultfontfeatures[\rmfamily]{Ligatures=TeX,Scale=1}
\fi
% use upquote if available, for straight quotes in verbatim environments
\IfFileExists{upquote.sty}{\usepackage{upquote}}{}
\IfFileExists{microtype.sty}{% use microtype if available
  \usepackage[]{microtype}
  \UseMicrotypeSet[protrusion]{basicmath} % disable protrusion for tt fonts
}{}
\makeatletter
\@ifundefined{KOMAClassName}{% if non-KOMA class
  \IfFileExists{parskip.sty}{%
    \usepackage{parskip}
  }{% else
    \setlength{\parindent}{0pt}
    \setlength{\parskip}{6pt plus 2pt minus 1pt}}
}{% if KOMA class
  \KOMAoptions{parskip=half}}
\makeatother
\usepackage{xcolor}
\IfFileExists{xurl.sty}{\usepackage{xurl}}{} % add URL line breaks if available
\IfFileExists{bookmark.sty}{\usepackage{bookmark}}{\usepackage{hyperref}}
\hypersetup{
  pdftitle={Principal Components \& Multiple Correspondence Analyses},
  pdfauthor={Derek Beaton},
  pdfborder={0 0 0},
  breaklinks=true}
\urlstyle{same}  % don't use monospace font for urls
\newif\ifbibliography
\usepackage{graphicx,grffile}
\makeatletter
\def\maxwidth{\ifdim\Gin@nat@width>\linewidth\linewidth\else\Gin@nat@width\fi}
\def\maxheight{\ifdim\Gin@nat@height>\textheight\textheight\else\Gin@nat@height\fi}
\makeatother
% Scale images if necessary, so that they will not overflow the page
% margins by default, and it is still possible to overwrite the defaults
% using explicit options in \includegraphics[width, height, ...]{}
\setkeys{Gin}{width=\maxwidth,height=\maxheight,keepaspectratio}
\setlength{\emergencystretch}{3em}  % prevent overfull lines
\providecommand{\tightlist}{%
  \setlength{\itemsep}{0pt}\setlength{\parskip}{0pt}}
\setcounter{secnumdepth}{-2}

% set default figure placement to htbp
\makeatletter
\def\fps@figure{htbp}
\makeatother

\usepackage{amssymb}
\usepackage{amsmath}
\usepackage{mathtools}
\usepackage{animate}
\usepackage{caption}
\captionsetup[figure]{labelformat=empty}
\usepackage{booktabs}
\usepackage{longtable}
\usepackage{array}
\usepackage{multirow}
\usepackage{wrapfig}
\usepackage{float}
\usepackage{colortbl}
\usepackage{pdflscape}
\usepackage{tabu}
\usepackage{threeparttable}
\AtBeginSubsection{}
\usepackage{textcomp}

\title{Principal Components \& Multiple Correspondence Analyses}
\subtitle{with resampling approaches for stability assessments}
\author{Derek Beaton}
\date{May 01, 2019}
\institute{RRI RTC}

\begin{document}
\frame{\titlepage}

\begin{frame}{Overview}
\protect\hypertarget{overview}{}

\begin{itemize}[<+->]
\tightlist
\item
  Something here
\item
  Where to find everything
\end{itemize}

\end{frame}

\hypertarget{introduction}{%
\section{Introduction}\label{introduction}}

\hypertarget{some-section}{%
\subsection{Some section}\label{some-section}}

\begin{frame}{Slide A}
\protect\hypertarget{slide-a}{}

Slide A

\end{frame}

\begin{frame}{Slide B}
\protect\hypertarget{slide-b}{}

Slide B

\end{frame}

\hypertarget{data}{%
\subsection{Data}\label{data}}

\begin{frame}{Taxonomy}
\protect\hypertarget{taxonomy}{}

\end{frame}

\begin{frame}{The actual data}
\protect\hypertarget{the-actual-data}{}

\end{frame}

\hypertarget{history}{%
\subsection{History}\label{history}}

\begin{frame}{PCA}
\protect\hypertarget{pca}{}

Stuff

\end{frame}

\begin{frame}{PCA}
\protect\hypertarget{pca-1}{}

How \& When to use it

\end{frame}

\begin{frame}{CA}
\protect\hypertarget{ca}{}

Stuff

\end{frame}

\begin{frame}{CA}
\protect\hypertarget{ca-1}{}

How \& When to use it

\end{frame}

\hypertarget{software}{%
\subsection{Software}\label{software}}

\begin{frame}{Today}
\protect\hypertarget{today}{}

\end{frame}

\begin{frame}{Some alternatives}
\protect\hypertarget{some-alternatives}{}

\end{frame}

\hypertarget{principal-components-analysis}{%
\section{Principal Components
Analysis}\label{principal-components-analysis}}

\hypertarget{fit}{%
\subsection{Fit}\label{fit}}

\begin{frame}

\includegraphics{test2_files/figure-beamer/unnamed-chunk-1-1.pdf}

\end{frame}

\begin{frame}

\includegraphics{test2_files/figure-beamer/unnamed-chunk-2-1.pdf}

\end{frame}

\begin{frame}

\includegraphics{test2_files/figure-beamer/unnamed-chunk-3-1.pdf}

\end{frame}

\begin{frame}

\includegraphics{test2_files/figure-beamer/unnamed-chunk-4-1.pdf}

\end{frame}

\begin{frame}

\includegraphics{test2_files/figure-beamer/unnamed-chunk-5-1.pdf}

\end{frame}

\begin{frame}

\includegraphics{test2_files/figure-beamer/unnamed-chunk-6-1.pdf}

\end{frame}

\begin{frame}

\includegraphics{test2_files/figure-beamer/unnamed-chunk-7-1.pdf}

\end{frame}

\begin{frame}

\includegraphics{test2_files/figure-beamer/unnamed-chunk-8-1.pdf}

\end{frame}

\begin{frame}{The SVD}
\protect\hypertarget{the-svd}{}

\end{frame}

\hypertarget{interpretation}{%
\subsection{Interpretation}\label{interpretation}}

\begin{frame}

\includegraphics{test2_files/figure-beamer/unnamed-chunk-9-1.pdf}

\end{frame}

\begin{frame}

\includegraphics{test2_files/figure-beamer/unnamed-chunk-10-1.pdf}

\end{frame}

\begin{frame}

\includegraphics{test2_files/figure-beamer/unnamed-chunk-11-1.pdf}

\end{frame}

\begin{frame}

\includegraphics{test2_files/figure-beamer/unnamed-chunk-12-1.pdf}

\end{frame}

\begin{frame}

\includegraphics{test2_files/figure-beamer/unnamed-chunk-13-1.pdf}

\end{frame}

\begin{frame}

\includegraphics{test2_files/figure-beamer/unnamed-chunk-14-1.pdf}

\end{frame}

\hypertarget{example}{%
\subsection{Example}\label{example}}

\begin{frame}{Scaling up}
\protect\hypertarget{scaling-up}{}

here: show the table and perhaps the code? maybe the code per slide for
easy slides?

\end{frame}

\begin{frame}

\includegraphics{test2_files/figure-beamer/unnamed-chunk-15-1.pdf}

\end{frame}

\begin{frame}

\includegraphics{test2_files/figure-beamer/unnamed-chunk-16-1.pdf}

\end{frame}

\begin{frame}

\includegraphics{test2_files/figure-beamer/unnamed-chunk-17-1.pdf}

\end{frame}

\begin{frame}

\includegraphics{test2_files/figure-beamer/unnamed-chunk-18-1.pdf}

\end{frame}

\begin{frame}

\includegraphics{test2_files/figure-beamer/unnamed-chunk-19-1.pdf}

\end{frame}

\begin{frame}

\includegraphics{test2_files/figure-beamer/unnamed-chunk-20-1.pdf}

\end{frame}

\hypertarget{correspondence-analyses}{%
\section{Correspondence analyses}\label{correspondence-analyses}}

\begin{frame}{}
\protect\hypertarget{section}{}

I don't know something.

\end{frame}

\hypertarget{small-examples}{%
\subsection{Small examples}\label{small-examples}}

\begin{frame}{Illustrative data}
\protect\hypertarget{illustrative-data}{}

\begin{table}[H]
\centering\begingroup\fontsize{10}{12}\selectfont

\begin{tabular}{lll}
\toprule
  & DX & PTRACCAT\\
\midrule
5023 & CN & Asian\\
5026 & MCI & White\\
5027 & Dementia & White\\
5028 & Dementia & White\\
5031 & MCI & White\\
\addlinespace
5037 & Dementia & Black\\
5040 & CN & Black\\
5047 & MCI & Black\\
5054 & Dementia & White\\
5058 & Dementia & Asian\\
5063 & Dementia & White\\
\bottomrule
\end{tabular}\endgroup{}
\end{table}

\end{frame}

\begin{frame}{Disjunctive data}
\protect\hypertarget{disjunctive-data}{}

\begin{table}[H]
\centering
\resizebox{\linewidth}{!}{
\begin{tabular}{lrrrrrrr}
\toprule
  & DX.MCI & DX.CN & DX.Dementia & PTRACCAT.White & PTRACCAT.Other & PTRACCAT.Black & PTRACCAT.Asian\\
\midrule
5023 & 0 & 1 & 0 & 0 & 0 & 0 & 1\\
5026 & 1 & 0 & 0 & 1 & 0 & 0 & 0\\
5027 & 0 & 0 & 1 & 1 & 0 & 0 & 0\\
5028 & 0 & 0 & 1 & 1 & 0 & 0 & 0\\
5031 & 1 & 0 & 0 & 1 & 0 & 0 & 0\\
\addlinespace
5037 & 0 & 0 & 1 & 0 & 0 & 1 & 0\\
5040 & 0 & 1 & 0 & 0 & 0 & 1 & 0\\
5047 & 1 & 0 & 0 & 0 & 0 & 1 & 0\\
5054 & 0 & 0 & 1 & 1 & 0 & 0 & 0\\
5058 & 0 & 0 & 1 & 0 & 0 & 0 & 1\\
5063 & 0 & 0 & 1 & 1 & 0 & 0 & 0\\
\bottomrule
\end{tabular}}
\end{table}

\end{frame}

\begin{frame}{Disjunctive data}
\protect\hypertarget{disjunctive-data-1}{}

\begin{table}[H]
\centering
\resizebox{\linewidth}{!}{
\begin{tabular}{lrrrrrrr}
\toprule
  & DX.MCI & DX.CN & DX.Dementia & PTRACCAT.White & PTRACCAT.Other & PTRACCAT.Black & PTRACCAT.Asian\\
\midrule
5023 & 0 & 1 & 0 & 0 & 0 & 0 & 1\\
5026 & 1 & 0 & 0 & 1 & 0 & 0 & 0\\
5027 & 0 & 0 & 1 & 1 & 0 & 0 & 0\\
5028 & 0 & 0 & 1 & 1 & 0 & 0 & 0\\
5031 & 1 & 0 & 0 & 1 & 0 & 0 & 0\\
\addlinespace
5037 & 0 & 0 & 1 & 0 & 0 & 1 & 0\\
5040 & 0 & 1 & 0 & 0 & 0 & 1 & 0\\
5047 & 1 & 0 & 0 & 0 & 0 & 1 & 0\\
5054 & 0 & 0 & 1 & 1 & 0 & 0 & 0\\
5058 & 0 & 0 & 1 & 0 & 0 & 0 & 1\\
5063 & 0 & 0 & 1 & 1 & 0 & 0 & 0\\
\bottomrule
\end{tabular}}
\end{table}

\begin{itemize}[<+->]
\tightlist
\item
  Row sums are total number of \emph{original} variables
\item
  Sum within a variable (e.g.~DX) is total number of rows
\item
  Sum of the table is rows \(\times\) columns
\end{itemize}

\end{frame}

\begin{frame}{A bad idea: PCA}
\protect\hypertarget{a-bad-idea-pca}{}

\end{frame}

\begin{frame}

\includegraphics{test2_files/figure-beamer/unnamed-chunk-24-1.pdf}

\end{frame}

\begin{frame}{Why is that a bad idea?}
\protect\hypertarget{why-is-that-a-bad-idea}{}

\includegraphics{test2_files/figure-beamer/unnamed-chunk-25-1.pdf}

\end{frame}

\begin{frame}{A better idea}
\protect\hypertarget{a-better-idea}{}

\begin{itemize}[<+->]
\tightlist
\item
  Correspondence analysis (CA)

  \begin{itemize}[<+->]
  \tightlist
  \item
    Think of it as a \(\chi^2\) PCA
  \end{itemize}
\item
  Deals with categories, counts (amongst others)
\item
  Row and column component scores exist on same scale

  \begin{itemize}[<+->]
  \tightlist
  \item
    CA is a \emph{bivariate} technique
  \end{itemize}
\end{itemize}

\end{frame}

\begin{frame}

\includegraphics{test2_files/figure-beamer/unnamed-chunk-27-1.pdf}

\end{frame}

\begin{frame}

\includegraphics{test2_files/figure-beamer/unnamed-chunk-28-1.pdf}

\end{frame}

\begin{frame}

\includegraphics{test2_files/figure-beamer/unnamed-chunk-29-1.pdf}

\end{frame}

\begin{frame}{Multiple correspondence analysis}
\protect\hypertarget{multiple-correspondence-analysis}{}

\begin{itemize}[<+->]
\tightlist
\item
  An extension of CA
\item
  Accomodates multiple categorical variables (CA only does 2)
\item
  Corrects the dimensionality
\item
  Has nearly magical properties (we'll see later)
\end{itemize}

\end{frame}

\begin{frame}

\includegraphics{test2_files/figure-beamer/unnamed-chunk-31-1.pdf}

\end{frame}

\begin{frame}

\includegraphics{test2_files/figure-beamer/unnamed-chunk-32-1.pdf}

\end{frame}

\begin{frame}

\includegraphics{test2_files/figure-beamer/unnamed-chunk-33-1.pdf}

\end{frame}

\begin{frame}

\includegraphics{test2_files/figure-beamer/unnamed-chunk-34-1.pdf}

\end{frame}

\begin{frame}

\includegraphics{test2_files/figure-beamer/unnamed-chunk-35-1.pdf}

\end{frame}

\begin{frame}{Why does it look like that?}
\protect\hypertarget{why-does-it-look-like-that}{}

\begin{table}[H]
\centering
\resizebox{\linewidth}{!}{
\begin{tabular}{rrrrrrr}
\toprule
DX.MCI & DX.CN & DX.Dementia & PTRACCAT.White & PTRACCAT.Other & PTRACCAT.Black & PTRACCAT.Asian\\
\midrule
1 & 0 & 0 & 1 & 0 & 0 & 0\\
1 & 0 & 0 & 0 & 1 & 0 & 0\\
1 & 0 & 0 & 0 & 0 & 1 & 0\\
0 & 1 & 0 & 0 & 1 & 0 & 0\\
1 & 0 & 0 & 0 & 0 & 0 & 1\\
\addlinespace
0 & 1 & 0 & 1 & 0 & 0 & 0\\
0 & 1 & 0 & 0 & 0 & 1 & 0\\
0 & 0 & 1 & 1 & 0 & 0 & 0\\
0 & 1 & 0 & 0 & 0 & 0 & 1\\
0 & 0 & 1 & 0 & 0 & 0 & 1\\
0 & 0 & 1 & 0 & 0 & 1 & 0\\
\bottomrule
\end{tabular}}
\end{table}

\end{frame}

\begin{frame}{Compare the results}
\protect\hypertarget{compare-the-results}{}

\begin{table}[H]
\centering
\resizebox{\linewidth}{!}{
\begin{tabular}{lrrrrr}
\toprule
  & PCA Comp. 1 & PCA Comp. 2 & PCA Comp. 3 & PCA Comp. 4 & PCA Comp. 5\\
\midrule
MCA Comp. 1 & 0.17 & -0.25 & 0.92 & 0.06 & -0.26\\
MCA Comp. 2 & -0.78 & 0.36 & 0.28 & -0.42 & 0.03\\
\bottomrule
\end{tabular}}
\end{table}

\begin{itemize}[<+->]
\tightlist
\item
  CA \& MCA produce identical results, except MCA:

  \begin{itemize}[<+->]
  \tightlist
  \item
    Drops components
  \item
    Corrects explained variance
  \end{itemize}
\end{itemize}

\end{frame}

\hypertarget{example-1}{%
\subsection{Example}\label{example-1}}

\begin{frame}{Scaling up}
\protect\hypertarget{scaling-up-1}{}

SHow the data here a bit

\end{frame}

\begin{frame}

\includegraphics{test2_files/figure-beamer/unnamed-chunk-39-1.pdf}

\end{frame}

\begin{frame}

\includegraphics{test2_files/figure-beamer/unnamed-chunk-40-1.pdf}

\end{frame}

\begin{frame}

\includegraphics{test2_files/figure-beamer/unnamed-chunk-41-1.pdf}

\end{frame}

\begin{frame}

\includegraphics{test2_files/figure-beamer/unnamed-chunk-42-1.pdf}

\end{frame}

\hypertarget{binomial-data}{%
\subsection{Binomial data}\label{binomial-data}}

\begin{verbatim}
## [1] "Corrections have failed. Original information must be used."
\end{verbatim}

\begin{frame}{A very important detour}
\protect\hypertarget{a-very-important-detour}{}

\begin{table}[H]
\centering\begingroup\fontsize{7}{9}\selectfont

\begin{tabular}{lll}
\toprule
  & PTGENDER & PTETHCAT\\
\midrule
5023 & Female & Not Hisp/Latino\\
5026 & Female & Not Hisp/Latino\\
5027 & Male & Not Hisp/Latino\\
5028 & Male & Not Hisp/Latino\\
5031 & Female & Hisp/Latino\\
\addlinespace
5037 & Male & Not Hisp/Latino\\
5040 & Female & Not Hisp/Latino\\
5047 & Female & Not Hisp/Latino\\
5054 & Female & Not Hisp/Latino\\
5058 & Male & Not Hisp/Latino\\
5063 & Female & Not Hisp/Latino\\
\bottomrule
\end{tabular}\endgroup{}
\end{table}

\begin{center}Two variables with strictly two levels (i.e., binary data)\end{center}

\end{frame}

\begin{frame}{A very important detour}
\protect\hypertarget{a-very-important-detour-1}{}

\begin{table}[H]
\centering
\resizebox{\linewidth}{!}{
\begin{tabular}{lrrrr}
\toprule
  & PTGENDER.Male & PTGENDER.Female & PTETHCAT.Not Hisp/Latino & PTETHCAT.Hisp/Latino\\
\midrule
5023 & 0 & 1 & 1 & 0\\
5026 & 0 & 1 & 1 & 0\\
5027 & 1 & 0 & 1 & 0\\
5028 & 1 & 0 & 1 & 0\\
5031 & 0 & 1 & 0 & 1\\
\addlinespace
5037 & 1 & 0 & 1 & 0\\
5040 & 0 & 1 & 1 & 0\\
5047 & 0 & 1 & 1 & 0\\
5054 & 0 & 1 & 1 & 0\\
5058 & 1 & 0 & 1 & 0\\
5063 & 0 & 1 & 1 & 0\\
\bottomrule
\end{tabular}}
\end{table}

\begin{center}Disjunctive coding of two variables with strictly two levels (i.e., binary data) into four columns\end{center}

\end{frame}

\begin{frame}{A very important detour}
\protect\hypertarget{a-very-important-detour-2}{}

\begin{table}[H]
\centering\begingroup\fontsize{7}{9}\selectfont

\begin{tabular}{lll}
\toprule
  & PTGENDER & PTETHCAT\\
\midrule
5023 & Female & Not Hisp/Latino\\
5026 & Female & Not Hisp/Latino\\
5027 & Male & Not Hisp/Latino\\
5028 & Male & Not Hisp/Latino\\
5031 & Female & Hisp/Latino\\
\addlinespace
5037 & Male & Not Hisp/Latino\\
5040 & Female & Not Hisp/Latino\\
5047 & Female & Not Hisp/Latino\\
5054 & Female & Not Hisp/Latino\\
5058 & Male & Not Hisp/Latino\\
5063 & Female & Not Hisp/Latino\\
\bottomrule
\end{tabular}\endgroup{}
\end{table}

\begin{center}Two variables with strictly two levels (i.e., binary data)\end{center}

\end{frame}

\begin{frame}{A very important detour}
\protect\hypertarget{a-very-important-detour-3}{}

\begin{table}[H]
\centering\begingroup\fontsize{7}{9}\selectfont

\begin{tabular}{lrr}
\toprule
  & PTGENDER & PTETHCAT\\
\midrule
5023 & 1 & 0\\
5026 & 1 & 0\\
5027 & 0 & 0\\
5028 & 0 & 0\\
5031 & 1 & 1\\
\addlinespace
5037 & 0 & 0\\
5040 & 1 & 0\\
5047 & 1 & 0\\
5054 & 1 & 0\\
5058 & 0 & 0\\
5063 & 1 & 0\\
\bottomrule
\end{tabular}\endgroup{}
\end{table}

\begin{center}Binary coding of two variables with strictly two levels (i.e., binary data) in two columns\end{center}

\end{frame}

\begin{frame}{A very important detour}
\protect\hypertarget{a-very-important-detour-4}{}

\begin{table}[H]
\centering\begingroup\fontsize{7}{9}\selectfont

\begin{tabular}{lrr}
\toprule
  & PTGENDER & PTETHCAT\\
\midrule
5023 & 0 & 1\\
5026 & 0 & 1\\
5027 & 1 & 1\\
5028 & 1 & 1\\
5031 & 0 & 0\\
\addlinespace
5037 & 1 & 1\\
5040 & 0 & 1\\
5047 & 0 & 1\\
5054 & 0 & 1\\
5058 & 1 & 1\\
5063 & 0 & 1\\
\bottomrule
\end{tabular}\endgroup{}
\end{table}

\begin{center}Alternate but equivalent binary coding of two variables with strictly two levels (i.e., binary data) in two columns\end{center}

\end{frame}

\begin{frame}{Always a bad idea?}
\protect\hypertarget{always-a-bad-idea}{}

\begin{itemize}[<+->]
\tightlist
\item
  MCA on the disjunctive coded data
\item
  PCA on the binary coded data
\end{itemize}

\end{frame}

\begin{frame}

\includegraphics{test2_files/figure-beamer/unnamed-chunk-49-1.pdf}

\end{frame}

\begin{frame}

\includegraphics{test2_files/figure-beamer/unnamed-chunk-50-1.pdf}

\begin{center}Oh, weird!\end{center}

\end{frame}

\begin{frame}

\includegraphics{test2_files/figure-beamer/unnamed-chunk-51-1.pdf}

\begin{center}Component 2 is "flipped" \\
We will revisit this
\end{center}

\end{frame}

\begin{frame}

\begin{table}[H]
\centering
\resizebox{\linewidth}{!}{
\begin{tabular}{lrr}
\toprule
  & PCA Comp. 1 & PCA Comp. 2\\
\midrule
MCA Comp. 1 & 1 & 0\\
MCA Comp. 2 & 0 & -1\\
\bottomrule
\end{tabular}}
\end{table}
\begin{center}Oh, double weird!\end{center}

\end{frame}

\begin{frame}{Let's get weird}
\protect\hypertarget{lets-get-weird}{}

\begin{table}[H]
\centering
\resizebox{\linewidth}{!}{
\begin{tabular}{lrr}
\toprule
  & PTGENDER & PTETHCAT\\
\midrule
PTGENDER & 1.00 & 0.06\\
PTETHCAT & 0.06 & 1.00\\
\bottomrule
\end{tabular}}
\end{table}

\end{frame}

\begin{frame}{Let's get weird}
\protect\hypertarget{lets-get-weird-1}{}

\begin{table}[H]
\centering
\resizebox{\linewidth}{!}{
\begin{tabular}{lrr}
\toprule
  & PTGENDER & PTETHCAT\\
\midrule
PTGENDER & 1.00 & 0.06\\
PTETHCAT & 0.06 & 1.00\\
\bottomrule
\end{tabular}}
\end{table}

\begin{itemize}[<+->]
\item
  \(\phi=\) 0.06
\item
  Deep connections between \(\chi^2\), Normal, binomial (and others)
\item
  We can expand the idea of ``binary'' or ``binomial''
\end{itemize}

\end{frame}

\hypertarget{continuous-data}{%
\subsection{Continuous data}\label{continuous-data}}

\begin{frame}{An old friend}
\protect\hypertarget{an-old-friend}{}

\begin{table}[H]
\centering\begingroup\fontsize{7}{9}\selectfont

\begin{tabular}{lrr}
\toprule
  & mPACCtrailsB & FDG\\
\midrule
5023 & 1.12 & 0.13\\
5026 & 0.46 & -1.31\\
5027 & -2.77 & -1.48\\
5028 & -1.59 & -0.97\\
5031 & -0.92 & -0.87\\
\addlinespace
5037 & -1.86 & -2.00\\
5040 & 0.94 & -0.21\\
5047 & -0.25 & 3.05\\
5054 & -0.80 & -1.05\\
5058 & -1.12 & -2.13\\
5063 & -2.31 & -2.49\\
\bottomrule
\end{tabular}\endgroup{}
\end{table}
\begin{center}We perform PCA on these data\end{center}

\end{frame}

\begin{frame}{Escofier's Geometric Magic}
\protect\hypertarget{escofiers-geometric-magic}{}

\begin{itemize}[<+->]
\tightlist
\item
  One of the ``fuzzy'' or ``bipolar'' coding schemes
\item
  Take each Z-scored continuous variable
\item
  Duplicate it as
  \(\begin{bmatrix} \frac{1-Z}{2} \frac{1+Z}{2}\end{bmatrix}\)
\end{itemize}

\end{frame}

\begin{frame}{Escofier's Geometric Magic}
\protect\hypertarget{escofiers-geometric-magic-1}{}

\begin{table}[H]
\centering\begingroup\fontsize{7}{9}\selectfont

\begin{tabular}{lrrrr}
\toprule
  & mPACCtrailsB- & mPACCtrailsB+ & FDG- & FDG+\\
\midrule
5023 & -0.06 & 1.06 & 0.43 & 0.57\\
5026 & 0.27 & 0.73 & 1.16 & -0.16\\
5027 & 1.88 & -0.88 & 1.24 & -0.24\\
5028 & 1.30 & -0.30 & 0.98 & 0.02\\
5031 & 0.96 & 0.04 & 0.93 & 0.07\\
\addlinespace
5037 & 1.43 & -0.43 & 1.50 & -0.50\\
5040 & 0.03 & 0.97 & 0.60 & 0.40\\
5047 & 0.62 & 0.38 & -1.03 & 2.03\\
5054 & 0.90 & 0.10 & 1.03 & -0.03\\
5058 & 1.06 & -0.06 & 1.57 & -0.57\\
5063 & 1.66 & -0.66 & 1.74 & -0.74\\
\bottomrule
\end{tabular}\endgroup{}
\end{table}

\end{frame}

\begin{frame}{Escofier's Geometric Magic}
\protect\hypertarget{escofiers-geometric-magic-2}{}

\begin{table}[H]
\centering\begingroup\fontsize{7}{9}\selectfont

\begin{tabular}{lrrrr}
\toprule
  & mPACCtrailsB- & mPACCtrailsB+ & FDG- & FDG+\\
\midrule
5023 & -0.06 & 1.06 & 0.43 & 0.57\\
5026 & 0.27 & 0.73 & 1.16 & -0.16\\
5027 & 1.88 & -0.88 & 1.24 & -0.24\\
5028 & 1.30 & -0.30 & 0.98 & 0.02\\
5031 & 0.96 & 0.04 & 0.93 & 0.07\\
\addlinespace
5037 & 1.43 & -0.43 & 1.50 & -0.50\\
5040 & 0.03 & 0.97 & 0.60 & 0.40\\
5047 & 0.62 & 0.38 & -1.03 & 2.03\\
5054 & 0.90 & 0.10 & 1.03 & -0.03\\
5058 & 1.06 & -0.06 & 1.57 & -0.57\\
5063 & 1.66 & -0.66 & 1.74 & -0.74\\
\bottomrule
\end{tabular}\endgroup{}
\end{table}

\begin{itemize}[<+->]
\tightlist
\item
  Row sums are total number of \emph{original} variables
\item
  Sum within a variable (e.g., FDG) is total number of rows
\item
  Sum of the table is rows \(\times\) columns
\end{itemize}

\end{frame}

\begin{frame}

\includegraphics{test2_files/figure-beamer/unnamed-chunk-60-1.pdf}

\begin{center}Oh, interesting!\\
Take note: each variable has two "poles"\end{center}

\end{frame}

\begin{frame}

\includegraphics{test2_files/figure-beamer/unnamed-chunk-61-1.pdf}

\begin{center}Oh, weird!\end{center}

\end{frame}

\begin{frame}

\includegraphics{test2_files/figure-beamer/unnamed-chunk-62-1.pdf}

\begin{center}Oh, double weird!\end{center}

\end{frame}

\begin{frame}

\includegraphics{test2_files/figure-beamer/unnamed-chunk-63-1.pdf}

\begin{center}Flips: They don't matter.\end{center}

\end{frame}

\begin{frame}

\begin{table}[H]
\centering
\resizebox{\linewidth}{!}{
\begin{tabular}{lrr}
\toprule
  & PCA Comp. 1 & PCA Comp. 2\\
\midrule
CA Comp. 1 & 1 & 0\\
CA Comp. 2 & 0 & -1\\
\bottomrule
\end{tabular}}
\end{table}
\begin{center}Flips: They don't matter.\end{center}

\end{frame}

\begin{frame}{Escofier's Geometric Trick}
\protect\hypertarget{escofiers-geometric-trick}{}

\begin{itemize}[<+->]
\tightlist
\item
  Apply PCA to continuous data or
\item
  Apply CA to ``Escofier transformed'' data
\end{itemize}

\end{frame}

\hypertarget{ordinal-data}{%
\subsection{Ordinal data}\label{ordinal-data}}

\begin{frame}{Thermometer}
\protect\hypertarget{thermometer}{}

\begin{itemize}[<+->]
\tightlist
\item
  For ordinal data
\item
  Another ``fuzzy'' or ``bipolar'' coding
\item
  More Escofier Geometric Magic

  \begin{itemize}[<+->]
  \tightlist
  \item
    Subtract the maximum (minimum is now \(0\))
  \item
    \(\begin{bmatrix} \frac{\texttt{max}(x)-x}{\texttt{max}} \frac{x-\texttt{min}(x)}{\texttt{max}}\end{bmatrix}\)
  \end{itemize}
\item
  Apply CA
\end{itemize}

\end{frame}

\begin{frame}{More Geometric Magic}
\protect\hypertarget{more-geometric-magic}{}

\begin{table}[H]
\centering\begingroup\fontsize{7}{9}\selectfont

\begin{tabular}{lrrrr}
\toprule
  & PTEDUCAT & CDRSB & ADAS13 & MOCA\\
\midrule
5023 & 18 & 0.0 & 6 & 30\\
5026 & 18 & 1.5 & 8 & 24\\
5027 & 18 & 4.0 & 27 & 19\\
5028 & 16 & 3.5 & 20 & 19\\
5031 & 14 & 2.0 & 16 & 20\\
\addlinespace
5037 & 16 & 5.0 & 35 & 17\\
5040 & 18 & 0.0 & 8 & 20\\
5047 & 16 & 1.0 & 17 & 24\\
5054 & 18 & 3.5 & 22 & 21\\
5058 & 20 & 3.0 & 17 & 21\\
5063 & 14 & 2.5 & 38 & 16\\
\bottomrule
\end{tabular}\endgroup{}
\end{table}

\end{frame}

\begin{frame}{More Geometric Magic}
\protect\hypertarget{more-geometric-magic-1}{}

\begin{table}[H]
\centering
\resizebox{\linewidth}{!}{
\begin{tabular}{lrrrrrrrr}
\toprule
  & PTEDUCAT+ & PTEDUCAT- & CDRSB+ & CDRSB- & ADAS13+ & ADAS13- & MOCA+ & MOCA-\\
\midrule
5023 & 0.75 & 0.25 & 0.00 & 1.00 & 0.13 & 0.87 & 1.00 & 0.00\\
5026 & 0.75 & 0.25 & 0.27 & 0.73 & 0.17 & 0.83 & 0.57 & 0.43\\
5027 & 0.75 & 0.25 & 0.73 & 0.27 & 0.59 & 0.41 & 0.21 & 0.79\\
5028 & 0.50 & 0.50 & 0.64 & 0.36 & 0.43 & 0.57 & 0.21 & 0.79\\
5031 & 0.25 & 0.75 & 0.36 & 0.64 & 0.35 & 0.65 & 0.29 & 0.71\\
\addlinespace
5037 & 0.50 & 0.50 & 0.91 & 0.09 & 0.76 & 0.24 & 0.07 & 0.93\\
5040 & 0.75 & 0.25 & 0.00 & 1.00 & 0.17 & 0.83 & 0.29 & 0.71\\
5047 & 0.50 & 0.50 & 0.18 & 0.82 & 0.37 & 0.63 & 0.57 & 0.43\\
5054 & 0.75 & 0.25 & 0.64 & 0.36 & 0.48 & 0.52 & 0.36 & 0.64\\
5058 & 1.00 & 0.00 & 0.55 & 0.45 & 0.37 & 0.63 & 0.36 & 0.64\\
5063 & 0.25 & 0.75 & 0.45 & 0.55 & 0.83 & 0.17 & 0.00 & 1.00\\
\bottomrule
\end{tabular}}
\end{table}

\end{frame}

\begin{frame}{More Geometric Magic}
\protect\hypertarget{more-geometric-magic-2}{}

\begin{table}[H]
\centering
\resizebox{\linewidth}{!}{
\begin{tabular}{lrrrrrrrr}
\toprule
  & PTEDUCAT+ & PTEDUCAT- & CDRSB+ & CDRSB- & ADAS13+ & ADAS13- & MOCA+ & MOCA-\\
\midrule
5023 & 0.75 & 0.25 & 0.00 & 1.00 & 0.13 & 0.87 & 1.00 & 0.00\\
5026 & 0.75 & 0.25 & 0.27 & 0.73 & 0.17 & 0.83 & 0.57 & 0.43\\
5027 & 0.75 & 0.25 & 0.73 & 0.27 & 0.59 & 0.41 & 0.21 & 0.79\\
5028 & 0.50 & 0.50 & 0.64 & 0.36 & 0.43 & 0.57 & 0.21 & 0.79\\
5031 & 0.25 & 0.75 & 0.36 & 0.64 & 0.35 & 0.65 & 0.29 & 0.71\\
\addlinespace
5037 & 0.50 & 0.50 & 0.91 & 0.09 & 0.76 & 0.24 & 0.07 & 0.93\\
5040 & 0.75 & 0.25 & 0.00 & 1.00 & 0.17 & 0.83 & 0.29 & 0.71\\
5047 & 0.50 & 0.50 & 0.18 & 0.82 & 0.37 & 0.63 & 0.57 & 0.43\\
5054 & 0.75 & 0.25 & 0.64 & 0.36 & 0.48 & 0.52 & 0.36 & 0.64\\
5058 & 1.00 & 0.00 & 0.55 & 0.45 & 0.37 & 0.63 & 0.36 & 0.64\\
5063 & 0.25 & 0.75 & 0.45 & 0.55 & 0.83 & 0.17 & 0.00 & 1.00\\
\bottomrule
\end{tabular}}
\end{table}

\begin{itemize}[<+->]
\tightlist
\item
  Row sums are total number of \emph{original} variables
\item
  Sum within a variable (e.g.~EDU) is total number of rows
\item
  Sum of the table is rows \(\times\) columns
\end{itemize}

\end{frame}

\begin{frame}

\includegraphics{test2_files/figure-beamer/unnamed-chunk-69-1.pdf}

\end{frame}

\begin{frame}

\includegraphics{test2_files/figure-beamer/unnamed-chunk-70-1.pdf}

\end{frame}

\begin{frame}

\includegraphics{test2_files/figure-beamer/unnamed-chunk-71-1.pdf}

\end{frame}

\begin{frame}

\includegraphics{test2_files/figure-beamer/unnamed-chunk-72-1.pdf}

\end{frame}

\begin{frame}

\includegraphics{test2_files/figure-beamer/unnamed-chunk-73-1.pdf}

\end{frame}

\begin{frame}

\includegraphics{test2_files/figure-beamer/unnamed-chunk-74-1.pdf}

\end{frame}

\hypertarget{ordinal-vs.-disjunctive}{%
\subsection{Ordinal vs.~Disjunctive}\label{ordinal-vs.-disjunctive}}

\begin{frame}{Thermometer vs.~Disjunctive}
\protect\hypertarget{thermometer-vs.-disjunctive}{}

\begin{itemize}[<+->]
\tightlist
\item
  Sometimes data could be either
\item
  Let's analyze it both ways
\end{itemize}

\end{frame}

\begin{frame}

\begin{table}[H]
\centering\begingroup\fontsize{7}{9}\selectfont

\begin{tabular}{lrr}
\toprule
  & APOE4 & HMSCORE\\
\midrule
5023 & 0 & 0\\
5026 & 1 & 1\\
5027 & 0 & 1\\
5028 & 2 & 1\\
5031 & 0 & 1\\
\addlinespace
5037 & 1 & 1\\
5040 & 0 & 1\\
5047 & 2 & 1\\
5054 & 1 & 0\\
5058 & 0 & 0\\
5063 & 1 & 1\\
\bottomrule
\end{tabular}\endgroup{}
\end{table}

\end{frame}

\begin{frame}

\includegraphics{test2_files/figure-beamer/unnamed-chunk-77-1.pdf}

\end{frame}

\begin{frame}

\includegraphics{test2_files/figure-beamer/unnamed-chunk-78-1.pdf}

\end{frame}

\begin{frame}{Thermometer vs.~Disjunctive}
\protect\hypertarget{thermometer-vs.-disjunctive-1}{}

\begin{itemize}[<+->]
\tightlist
\item
  For a small (reasonable) number of levels: disjunctive
\item
  Otherwise: thermometer
\item
  Interpretation:

  \begin{itemize}[<+->]
  \tightlist
  \item
    Thermometer is ``easier''
  \item
    Disjunctive is more informative
  \end{itemize}
\end{itemize}

\end{frame}

\hypertarget{mixed-data}{%
\subsection{Mixed data}\label{mixed-data}}

\begin{frame}{All of the data}
\protect\hypertarget{all-of-the-data}{}

\begin{table}[H]
\centering
\resizebox{\linewidth}{!}{
\begin{tabular}{llrlrllrrrrrrrrrrr}
\toprule
  & DX & AGE & PTGENDER & PTEDUCAT & PTETHCAT & PTRACCAT & APOE4 & FDG & AV45 & CDRSB & ADAS13 & MOCA & WholeBrain & Hippocampus & MidTemp & mPACCtrailsB & HMSCORE\\
\midrule
5023 & CN & 63.9 & Female & 18 & Not Hisp/Latino & Asian & 0 & 1.29 & 1.03 & 0.0 & 6 & 30 & 1057351.0 & 7904 & 21306 & 1.81 & 0\\
5026 & MCI & 70.5 & Female & 18 & Not Hisp/Latino & White & 1 & 1.08 & 1.44 & 1.5 & 8 & 24 & 1023057.3 & 8051 & 16501 & -1.45 & 1\\
5027 & Dementia & 75.5 & Male & 18 & Not Hisp/Latino & White & 0 & 1.06 & 1.44 & 4.0 & 27 & 19 & 986723.7 & 6534 & 17437 & -17.27 & 1\\
5028 & Dementia & 61.9 & Male & 16 & Not Hisp/Latino & White & 2 & 1.13 & 1.38 & 3.5 & 20 & 19 & 1182704.6 & 7481 & 20797 & -11.50 & 1\\
5031 & MCI & 80.2 & Female & 14 & Hisp/Latino & White & 0 & 1.14 & 1.52 & 2.0 & 16 & 20 & 908133.9 & 5040 & 19032 & -8.21 & 1\\
\addlinespace
5037 & Dementia & 67.3 & Male & 16 & Not Hisp/Latino & Black & 1 & 0.98 & 1.21 & 5.0 & 35 & 17 & 1161499.6 & 5831 & 21428 & -12.80 & 1\\
5040 & CN & 75.9 & Female & 18 & Not Hisp/Latino & Black & 0 & 1.24 & 1.01 & 0.0 & 8 & 20 & 943160.6 & 7994 & 16634 & 0.94 & 1\\
5047 & MCI & 68.8 & Female & 16 & Not Hisp/Latino & Black & 2 & 1.70 & 1.48 & 1.0 & 17 & 24 & 1070406.1 & 7920 & 22043 & -4.90 & 1\\
5054 & Dementia & 74.0 & Female & 18 & Not Hisp/Latino & White & 1 & 1.12 & 1.43 & 3.5 & 22 & 21 & 1138040.1 & 6580 & 20836 & -7.63 & 0\\
5058 & Dementia & 61.8 & Male & 20 & Not Hisp/Latino & Asian & 0 & 0.97 & 1.54 & 3.0 & 17 & 21 & 1195549.3 & 7318 & 22757 & -9.18 & 0\\
5063 & Dementia & 71.5 & Female & 14 & Not Hisp/Latino & White & 1 & 0.92 & 1.61 & 2.5 & 38 & 16 & 817421.2 & 5364 & 12542 & -15.03 & 1\\
\bottomrule
\end{tabular}}
\end{table}

\end{frame}

\begin{frame}

\includegraphics{test2_files/figure-beamer/unnamed-chunk-81-1.pdf}

\end{frame}

\begin{frame}

\includegraphics{test2_files/figure-beamer/unnamed-chunk-82-1.pdf}

\end{frame}

\hypertarget{resampling}{%
\section{Resampling}\label{resampling}}

\hypertarget{permutation}{%
\subsection{Permutation}\label{permutation}}

\hypertarget{bootstrap}{%
\subsection{Bootstrap}\label{bootstrap}}

\hypertarget{split-half}{%
\subsection{Split-half}\label{split-half}}

\begin{frame}{FUCK}
\protect\hypertarget{fuck}{}

\end{frame}

\end{document}

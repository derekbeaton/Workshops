\PassOptionsToPackage{unicode=true}{hyperref} % options for packages loaded elsewhere
\PassOptionsToPackage{hyphens}{url}
%
\documentclass[
  ignorenonframetext,
]{beamer}
\usepackage{pgfpages}
\setbeamertemplate{caption}[numbered]
\setbeamertemplate{caption label separator}{: }
\setbeamercolor{caption name}{fg=normal text.fg}
\beamertemplatenavigationsymbolsempty
% Prevent slide breaks in the middle of a paragraph:
\widowpenalties 1 10000
\raggedbottom
\setbeamertemplate{part page}{
  \centering
  \begin{beamercolorbox}[sep=16pt,center]{part title}
    \usebeamerfont{part title}\insertpart\par
  \end{beamercolorbox}
}
\setbeamertemplate{section page}{
  \centering
  \begin{beamercolorbox}[sep=12pt,center]{part title}
    \usebeamerfont{section title}\insertsection\par
  \end{beamercolorbox}
}
\setbeamertemplate{subsection page}{
  \centering
  \begin{beamercolorbox}[sep=8pt,center]{part title}
    \usebeamerfont{subsection title}\insertsubsection\par
  \end{beamercolorbox}
}
\AtBeginPart{
  \frame{\partpage}
}
\AtBeginSection{
  \ifbibliography
  \else
    \frame{\sectionpage}
  \fi
}
\AtBeginSubsection{
  \frame{\subsectionpage}
}
\usepackage{lmodern}
\usepackage{amssymb,amsmath}
\usepackage{ifxetex,ifluatex}
\ifnum 0\ifxetex 1\fi\ifluatex 1\fi=0 % if pdftex
  \usepackage[T1]{fontenc}
  \usepackage[utf8]{inputenc}
  \usepackage{textcomp} % provides euro and other symbols
\else % if luatex or xelatex
  \usepackage{unicode-math}
  \defaultfontfeatures{Scale=MatchLowercase}
  \defaultfontfeatures[\rmfamily]{Ligatures=TeX,Scale=1}
\fi
% use upquote if available, for straight quotes in verbatim environments
\IfFileExists{upquote.sty}{\usepackage{upquote}}{}
\IfFileExists{microtype.sty}{% use microtype if available
  \usepackage[]{microtype}
  \UseMicrotypeSet[protrusion]{basicmath} % disable protrusion for tt fonts
}{}
\makeatletter
\@ifundefined{KOMAClassName}{% if non-KOMA class
  \IfFileExists{parskip.sty}{%
    \usepackage{parskip}
  }{% else
    \setlength{\parindent}{0pt}
    \setlength{\parskip}{6pt plus 2pt minus 1pt}}
}{% if KOMA class
  \KOMAoptions{parskip=half}}
\makeatother
\usepackage{xcolor}
\IfFileExists{xurl.sty}{\usepackage{xurl}}{} % add URL line breaks if available
\IfFileExists{bookmark.sty}{\usepackage{bookmark}}{\usepackage{hyperref}}
\hypersetup{
  pdftitle={Principal Components \& Multiple Correspondence Analyses},
  pdfauthor={Derek Beaton},
  pdfborder={0 0 0},
  breaklinks=true}
\urlstyle{same}  % don't use monospace font for urls
\newif\ifbibliography
\usepackage{graphicx,grffile}
\makeatletter
\def\maxwidth{\ifdim\Gin@nat@width>\linewidth\linewidth\else\Gin@nat@width\fi}
\def\maxheight{\ifdim\Gin@nat@height>\textheight\textheight\else\Gin@nat@height\fi}
\makeatother
% Scale images if necessary, so that they will not overflow the page
% margins by default, and it is still possible to overwrite the defaults
% using explicit options in \includegraphics[width, height, ...]{}
\setkeys{Gin}{width=\maxwidth,height=\maxheight,keepaspectratio}
\setlength{\emergencystretch}{3em}  % prevent overfull lines
\providecommand{\tightlist}{%
  \setlength{\itemsep}{0pt}\setlength{\parskip}{0pt}}
\setcounter{secnumdepth}{-2}

% set default figure placement to htbp
\makeatletter
\def\fps@figure{htbp}
\makeatother

\usepackage{amssymb}
\usepackage{amsmath}
\usepackage{mathtools}
\usepackage{animate}
\usepackage{caption}
\captionsetup[figure]{labelformat=empty}
\usepackage{booktabs}
\usepackage{longtable}
\usepackage{array}
\usepackage{multirow}
\usepackage{wrapfig}
\usepackage{float}
\usepackage{colortbl}
\usepackage{pdflscape}
\usepackage{tabu}
\usepackage{threeparttable}
\AtBeginSubsection{}
\usepackage{textcomp}

\title{Principal Components \& Multiple Correspondence Analyses}
\subtitle{with resampling approaches for stability assessments}
\author{Derek Beaton}
\date{May 03, 2019}
\institute{RRI RTC}

\begin{document}
\frame{\titlepage}

\begin{frame}{Where to find everything}
\protect\hypertarget{where-to-find-everything}{}

\begin{itemize}[<+->]
\tightlist
\item
  Generally: \url{https://github.com/derekbeaton/workshops}
\item
  Today:
  \url{https://github.com/derekbeaton/Workshops/tree/master/RTC/PCA_MCA_Resampling}
\end{itemize}

\end{frame}

\begin{frame}{Some set up}
\protect\hypertarget{some-set-up}{}

\begin{itemize}[<+->]
\tightlist
\item
  Use RStudio (makes it easy)
\item
  You can pull from the Git repo

  \begin{itemize}[<+->]
  \tightlist
  \item
    Or copy individual files
  \end{itemize}
\item
  Make .Renviron file

  \begin{itemize}[<+->]
  \tightlist
  \item
    Points to locations outside the repo
  \end{itemize}
\item
  Run ``/R/0\_Create\_ADNI\_Dataset.R'' first

  \begin{itemize}[<+->]
  \tightlist
  \item
    Then either run this .Rmd or
  \item
    Run scripts in order
  \end{itemize}
\item
  Use of the ADNI data

  \begin{itemize}[<+->]
  \tightlist
  \item
    Via the `ADNIMERGE' package
  \end{itemize}
\end{itemize}

\end{frame}

\begin{frame}{An advertisement}
\protect\hypertarget{an-advertisement}{}

\begin{itemize}[<+->]
\tightlist
\item
  Lots of really cool R \& RStudio stuff
\item
  This presentation is 90\% reproducible

  \begin{itemize}[<+->]
  \tightlist
  \item
    Resampling is painful
  \end{itemize}
\item
  R \& RStudio ``Magic'' BrainHackTO tutorial

  \begin{itemize}[<+->]
  \tightlist
  \item
    Jenny Rieck \& I
  \item
    May 21 or 22
  \item
    Possibly sold out?
  \item
    We'll make stuff available
  \end{itemize}
\end{itemize}

\end{frame}

\begin{frame}{Motivation for today}
\protect\hypertarget{motivation-for-today}{}

\begin{itemize}[<+->]
\tightlist
\item
  Not everything is a number
\item
  But with care, it can be turned into one
\end{itemize}

\end{frame}

\begin{frame}{Overview}
\protect\hypertarget{overview}{}

\begin{itemize}[<+->]
\tightlist
\item
  Introduction
\item
  PCA
\item
  CA
\item
  Resampling
\item
  Final notes
\end{itemize}

\end{frame}

\hypertarget{introduction}{%
\section{Introduction}\label{introduction}}

\hypertarget{history}{%
\subsection{History}\label{history}}

\begin{frame}{Prehistory}
\protect\hypertarget{prehistory}{}

\begin{itemize}[<+->]
\tightlist
\item
  Basis:

  \begin{itemize}[<+->]
  \tightlist
  \item
    Hotelling (1933)
  \item
    Eckart \& Yong (1936)
  \end{itemize}
\item
  Traces back to

  \begin{itemize}[<+->]
  \tightlist
  \item
    Cauchy (1829)
  \item
    Galton (1859)
  \item
    K. Pearson (1901)
  \item
    Spearman (1904)
  \end{itemize}
\end{itemize}

\end{frame}

\begin{frame}{History}
\protect\hypertarget{history-1}{}

\begin{itemize}[<+->]
\tightlist
\item
  ``Modern form'' of PCA \& factor analyses

  \begin{itemize}[<+->]
  \tightlist
  \item
    Thurstone (1934)
  \item
    Fisher (1940)
  \item
    Tucker (too many to list)
  \item
    Many others in 1940s-1960s
  \end{itemize}
\item
  CA

  \begin{itemize}[<+->]
  \tightlist
  \item
    Hirschfeld (1935)
  \item
    Guttman (1941)
  \item
    Burt (1950)
  \item
    Benzecri (1964)
  \item
    Escofier (1965)
  \end{itemize}
\item
  See Lebart's History \& Prehistory of CA:
  \url{http://www.dtmvic.com/doc/About_the_History_of_CA.pdf}
\end{itemize}

\end{frame}

\begin{frame}{Now \& The Future}
\protect\hypertarget{now-the-future}{}

\begin{itemize}[<+->]
\tightlist
\item
  PCA is always cool.
\item
  See the final slides for related methods

  \begin{itemize}[<+->]
  \tightlist
  \item
    PCA makes you familiar with all of them
  \item
    CA makes you an expert with all of them
  \end{itemize}
\end{itemize}

\end{frame}

\hypertarget{pca-ca}{%
\subsection{PCA \& CA}\label{pca-ca}}

\begin{frame}{PCA \& CA}
\protect\hypertarget{pca-ca-1}{}

\begin{itemize}[<+->]
\tightlist
\item
  Visualize multiple/high dimensions
\item
  Dimensionality reduction
\item
  Matrix factorization
\item
  Unsupervised learning
\end{itemize}

\end{frame}

\begin{frame}{PCA \& CA}
\protect\hypertarget{pca-ca-2}{}

\begin{itemize}[<+->]
\tightlist
\item
  Find ``components''

  \begin{itemize}[<+->]
  \tightlist
  \item
    Components are new variables that are combinations of the original
    variables
  \end{itemize}
\item
  Components explain maximal variance

  \begin{itemize}[<+->]
  \tightlist
  \item
    Conditional to orthogonality
  \end{itemize}
\item
  So what's the difference?
\end{itemize}

\end{frame}

\begin{frame}{PCA vs CA}
\protect\hypertarget{pca-vs-ca}{}

\begin{itemize}[<+->]
\tightlist
\item
  PCA: For generally continuous (interval scale) data
\item
  CA: For (almost) everything else

  \begin{itemize}[<+->]
  \tightlist
  \item
    And also for continuous data!
  \end{itemize}
\end{itemize}

\end{frame}

\begin{frame}{Under the hood}
\protect\hypertarget{under-the-hood}{}

\begin{itemize}[<+->]
\tightlist
\item
  The eigenvalue decomposition (EVD)

  \begin{itemize}[<+->]
  \tightlist
  \item
    Requires squares, symmetric, and positive semi definite
  \item
    Generally correlation or covariance
  \end{itemize}
\item
  The singular value decomposition (SVD)

  \begin{itemize}[<+->]
  \tightlist
  \item
    Works with rectangular tables
  \end{itemize}
\item
  A generalized SVD

  \begin{itemize}[<+->]
  \tightlist
  \item
    Apply constraints (weights) to rows \& columns of rectangular table
  \item
    Required for CA and fancier PCA-like techniques \& extensions
  \end{itemize}
\end{itemize}

\end{frame}

\hypertarget{definitions}{%
\subsection{Definitions}\label{definitions}}

\begin{frame}{Terms I will use today}
\protect\hypertarget{terms-i-will-use-today}{}

\begin{itemize}[<+->]
\tightlist
\item
  Component scores

  \begin{itemize}[<+->]
  \tightlist
  \item
    Values assigned to rows (PCA \& CA) or columns (CA) scaled by
    variance
  \end{itemize}
\item
  Correlation loadings (PCA)

  \begin{itemize}[<+->]
  \tightlist
  \item
    Correlation of original data with row component scores
    (observations)
  \end{itemize}
\item
  Explained variance

  \begin{itemize}[<+->]
  \tightlist
  \item
    Eigenvalues
  \item
    How much of the total variance per component
  \item
    Variance = Sums of squares
  \end{itemize}
\item
  Magic
\end{itemize}

\end{frame}

\hypertarget{software}{%
\subsection{Software}\label{software}}

\begin{frame}{Today}
\protect\hypertarget{today}{}

\begin{itemize}[<+->]
\tightlist
\item
  ExPosition

  \begin{itemize}[<+->]
  \tightlist
  \item
    Family of packages
  \item
    Includes resampling
  \item
    Lots of PCA \& CA techniques
  \end{itemize}
\item
  factoextra

  \begin{itemize}[<+->]
  \tightlist
  \item
    Awesome ggplot2 visualizers for ExPosition
  \item
    \url{http://www.alboukadel.com/} \&
    \url{http://www.sthda.com/english/}
  \end{itemize}
\item
  ggplot2 \& tidyverse
\item
  ours

  \begin{itemize}[<+->]
  \tightlist
  \item
    Developed here within ONDRI
  \item
    New package for outliers
  \item
    Has some important bells-and-whistles
  \end{itemize}
\item
  Making things look fancy:

  \begin{itemize}[<+->]
  \tightlist
  \item
    kable, kableExtra, gridExtra, ggcorrplot
  \end{itemize}
\end{itemize}

\end{frame}

\begin{frame}{Some alternatives}
\protect\hypertarget{some-alternatives}{}

\begin{itemize}[<+->]
\tightlist
\item
  FactoMineR
\item
  ade4
\item
  ca
\item
  MASS
\item
  psych
\item
  So many others
\end{itemize}

\end{frame}

\hypertarget{data}{%
\subsection{Data}\label{data}}

\begin{frame}{Typology}
\protect\hypertarget{typology}{}

\begin{itemize}[<+->]
\tightlist
\item
  SS Stevens

  \begin{itemize}[<+->]
  \tightlist
  \item
    Not a boat!
  \end{itemize}
\item
  Levels of measurement

  \begin{itemize}[<+->]
  \tightlist
  \item
    Nominal (categorical)
  \item
    Ordinal (ranked, discrete categories)
  \item
    Interval (continuous, arbitrary 0)
  \item
    Ratio (continuous, non-arbitrary 0)
  \end{itemize}
\item
  Excellent examples:
  \url{https://en.wikipedia.org/wiki/Level_of_measurement}
\end{itemize}

\end{frame}

\begin{frame}{Today's data}
\protect\hypertarget{todays-data}{}

\begin{itemize}[<+->]
\tightlist
\item
  Alzheimer's Disease Neuroimaging Initiative (ADNI)
\item
  Data set:

  \begin{itemize}[<+->]
  \tightlist
  \item
    665 observations
  \item
    17 variables
  \end{itemize}
\item
  Walk through this set to tell a whole story
\end{itemize}

\end{frame}

\begin{frame}{Today's data}
\protect\hypertarget{todays-data-1}{}

\includegraphics{PCA_MCA_Slides_files/figure-beamer/unnamed-chunk-1-1.pdf}

\end{frame}

\hypertarget{principal-components-analysis}{%
\section{Principal Components
Analysis}\label{principal-components-analysis}}

\hypertarget{find-a-component}{%
\subsection{Find a component}\label{find-a-component}}

\begin{frame}{Let's dive in}
\protect\hypertarget{lets-dive-in}{}

\begin{itemize}[<+->]
\tightlist
\item
  We'll start with just two variables:
\item
  Trails

  \begin{itemize}[<+->]
  \tightlist
  \item
    Neuropsych test
  \item
    Executive function
  \end{itemize}
\item
  FDG

  \begin{itemize}[<+->]
  \tightlist
  \item
    PET imaging; brain function
  \item
    Average of several brain regions
  \end{itemize}
\end{itemize}

\end{frame}

\begin{frame}

\includegraphics{PCA_MCA_Slides_files/figure-beamer/unnamed-chunk-2-1.pdf}

\end{frame}

\begin{frame}

\includegraphics{PCA_MCA_Slides_files/figure-beamer/unnamed-chunk-3-1.pdf}

\end{frame}

\begin{frame}

\includegraphics{PCA_MCA_Slides_files/figure-beamer/unnamed-chunk-4-1.pdf}

\end{frame}

\begin{frame}

\includegraphics{PCA_MCA_Slides_files/figure-beamer/unnamed-chunk-5-1.pdf}

\end{frame}

\begin{frame}

\includegraphics{PCA_MCA_Slides_files/figure-beamer/unnamed-chunk-6-1.pdf}

\end{frame}

\begin{frame}

\includegraphics{PCA_MCA_Slides_files/figure-beamer/unnamed-chunk-7-1.pdf}

\end{frame}

\begin{frame}

\includegraphics{PCA_MCA_Slides_files/figure-beamer/unnamed-chunk-8-1.pdf}

\end{frame}

\begin{frame}

\includegraphics{PCA_MCA_Slides_files/figure-beamer/unnamed-chunk-9-1.pdf}

\end{frame}

\hypertarget{interpretation}{%
\subsection{Interpretation}\label{interpretation}}

\begin{frame}

\includegraphics{PCA_MCA_Slides_files/figure-beamer/unnamed-chunk-10-1.pdf}

\end{frame}

\begin{frame}

\includegraphics{PCA_MCA_Slides_files/figure-beamer/unnamed-chunk-11-1.pdf}

\end{frame}

\begin{frame}

\includegraphics{PCA_MCA_Slides_files/figure-beamer/unnamed-chunk-12-1.pdf}

\end{frame}

\begin{frame}

\includegraphics{PCA_MCA_Slides_files/figure-beamer/unnamed-chunk-13-1.pdf}

\end{frame}

\begin{frame}

\includegraphics{PCA_MCA_Slides_files/figure-beamer/unnamed-chunk-14-1.pdf}

\end{frame}

\begin{frame}

\includegraphics{PCA_MCA_Slides_files/figure-beamer/unnamed-chunk-15-1.pdf}

\end{frame}

\hypertarget{example}{%
\subsection{Example}\label{example}}

\begin{frame}{Scaling up}
\protect\hypertarget{scaling-up}{}

\begin{itemize}[<+->]
\tightlist
\item
  Scale up: MORE DATA!
\item
  All of the continuous variables
\end{itemize}

\end{frame}

\begin{frame}{Scaling up}
\protect\hypertarget{scaling-up-1}{}

\includegraphics{PCA_MCA_Slides_files/figure-beamer/unnamed-chunk-16-1.pdf}

\end{frame}

\begin{frame}{A new plot}
\protect\hypertarget{a-new-plot}{}

\begin{itemize}[<+->]
\tightlist
\item
  Scree (Cattell)
\item
  Junk at the bottom of a slope
\item
  Shows us explained variance (\%) per component
\end{itemize}

\end{frame}

\begin{frame}

\includegraphics{PCA_MCA_Slides_files/figure-beamer/unnamed-chunk-17-1.pdf}

\end{frame}

\begin{frame}

\includegraphics{PCA_MCA_Slides_files/figure-beamer/unnamed-chunk-18-1.pdf}

\end{frame}

\begin{frame}

\includegraphics{PCA_MCA_Slides_files/figure-beamer/unnamed-chunk-19-1.pdf}

\end{frame}

\begin{frame}

\includegraphics{PCA_MCA_Slides_files/figure-beamer/unnamed-chunk-20-1.pdf}

\end{frame}

\begin{frame}

\includegraphics{PCA_MCA_Slides_files/figure-beamer/unnamed-chunk-21-1.pdf}

\end{frame}

\begin{frame}

\includegraphics{PCA_MCA_Slides_files/figure-beamer/unnamed-chunk-22-1.pdf}

\end{frame}

\hypertarget{correspondence-analyses}{%
\section{Correspondence analyses}\label{correspondence-analyses}}

\begin{frame}{CA}
\protect\hypertarget{ca}{}

\begin{itemize}[<+->]
\tightlist
\item
  Like PCA in many ways
\item
  Slightly different interpretations
\item
  So much cooler

  \begin{itemize}[<+->]
  \tightlist
  \item
    Handles all types of data
  \end{itemize}
\end{itemize}

\end{frame}

\hypertarget{small-examples}{%
\subsection{Small examples}\label{small-examples}}

\begin{frame}{Illustrative data}
\protect\hypertarget{illustrative-data}{}

\begin{table}[H]
\centering\begingroup\fontsize{10}{12}\selectfont

\begin{tabular}{lll}
\toprule
  & DX & PTRACCAT\\
\midrule
5023 & CN & Asian\\
5026 & MCI & White\\
5027 & Dementia & White\\
5028 & Dementia & White\\
5031 & MCI & White\\
\addlinespace
5037 & Dementia & Black\\
5040 & CN & Black\\
5047 & MCI & Black\\
5054 & Dementia & White\\
5058 & Dementia & Asian\\
5063 & Dementia & White\\
\bottomrule
\end{tabular}\endgroup{}
\end{table}

\end{frame}

\begin{frame}{Disjunctive data}
\protect\hypertarget{disjunctive-data}{}

\begin{table}[H]
\centering
\resizebox{\linewidth}{!}{
\begin{tabular}{lrrrrrrr}
\toprule
  & DX.MCI & DX.CN & DX.Dementia & PTRACCAT.White & PTRACCAT.Other & PTRACCAT.Black & PTRACCAT.Asian\\
\midrule
5023 & 0 & 1 & 0 & 0 & 0 & 0 & 1\\
5026 & 1 & 0 & 0 & 1 & 0 & 0 & 0\\
5027 & 0 & 0 & 1 & 1 & 0 & 0 & 0\\
5028 & 0 & 0 & 1 & 1 & 0 & 0 & 0\\
5031 & 1 & 0 & 0 & 1 & 0 & 0 & 0\\
\addlinespace
5037 & 0 & 0 & 1 & 0 & 0 & 1 & 0\\
5040 & 0 & 1 & 0 & 0 & 0 & 1 & 0\\
5047 & 1 & 0 & 0 & 0 & 0 & 1 & 0\\
5054 & 0 & 0 & 1 & 1 & 0 & 0 & 0\\
5058 & 0 & 0 & 1 & 0 & 0 & 0 & 1\\
5063 & 0 & 0 & 1 & 1 & 0 & 0 & 0\\
\bottomrule
\end{tabular}}
\end{table}

\end{frame}

\begin{frame}{Disjunctive data}
\protect\hypertarget{disjunctive-data-1}{}

\begin{table}[H]
\centering
\resizebox{\linewidth}{!}{
\begin{tabular}{lrrrrrrr}
\toprule
  & DX.MCI & DX.CN & DX.Dementia & PTRACCAT.White & PTRACCAT.Other & PTRACCAT.Black & PTRACCAT.Asian\\
\midrule
5023 & 0 & 1 & 0 & 0 & 0 & 0 & 1\\
5026 & 1 & 0 & 0 & 1 & 0 & 0 & 0\\
5027 & 0 & 0 & 1 & 1 & 0 & 0 & 0\\
5028 & 0 & 0 & 1 & 1 & 0 & 0 & 0\\
5031 & 1 & 0 & 0 & 1 & 0 & 0 & 0\\
\addlinespace
5037 & 0 & 0 & 1 & 0 & 0 & 1 & 0\\
5040 & 0 & 1 & 0 & 0 & 0 & 1 & 0\\
5047 & 1 & 0 & 0 & 0 & 0 & 1 & 0\\
5054 & 0 & 0 & 1 & 1 & 0 & 0 & 0\\
5058 & 0 & 0 & 1 & 0 & 0 & 0 & 1\\
5063 & 0 & 0 & 1 & 1 & 0 & 0 & 0\\
\bottomrule
\end{tabular}}
\end{table}

\begin{itemize}[<+->]
\tightlist
\item
  Row sums are total number of \emph{original} variables
\item
  Sum within a variable (e.g.~DX) is total number of rows
\item
  Sum of the table is rows \(\times\) columns
\end{itemize}

\end{frame}

\begin{frame}{A bad idea: PCA}
\protect\hypertarget{a-bad-idea-pca}{}

\begin{itemize}[<+->]
\tightlist
\item
  ``coding categorical variables with the indicator matrix of dummy
  variables and considering them as Gaussian, for instance, is almost a
  crime.''

  \begin{itemize}[<+->]
  \tightlist
  \item
    ``Jan de Leeuw and the French School of Data Analysis'' (Husson,
    Josse, Saporta)
  \end{itemize}
\end{itemize}

\end{frame}

\begin{frame}

\includegraphics{PCA_MCA_Slides_files/figure-beamer/unnamed-chunk-26-1.pdf}

\end{frame}

\begin{frame}{Why is that a bad idea?}
\protect\hypertarget{why-is-that-a-bad-idea}{}

\includegraphics{PCA_MCA_Slides_files/figure-beamer/unnamed-chunk-27-1.pdf}

\end{frame}

\begin{frame}{A better idea}
\protect\hypertarget{a-better-idea}{}

\begin{itemize}[<+->]
\tightlist
\item
  Correspondence analysis (CA)

  \begin{itemize}[<+->]
  \tightlist
  \item
    Think of it as a \(\chi^2\) PCA
  \end{itemize}
\item
  Designed to handle things that look like counts

  \begin{itemize}[<+->]
  \tightlist
  \item
    That includes categories
  \item
    And some other things
  \end{itemize}
\item
  Row and column component scores exist on same scale

  \begin{itemize}[<+->]
  \tightlist
  \item
    CA is a \emph{bivariate} technique
  \end{itemize}
\end{itemize}

\end{frame}

\begin{frame}

\includegraphics{PCA_MCA_Slides_files/figure-beamer/unnamed-chunk-29-1.pdf}

\end{frame}

\begin{frame}

\includegraphics{PCA_MCA_Slides_files/figure-beamer/unnamed-chunk-30-1.pdf}

\end{frame}

\begin{frame}

\includegraphics{PCA_MCA_Slides_files/figure-beamer/unnamed-chunk-31-1.pdf}

\end{frame}

\begin{frame}{Multiple correspondence analysis}
\protect\hypertarget{multiple-correspondence-analysis}{}

\begin{itemize}[<+->]
\tightlist
\item
  An extension of CA
\item
  Accomodates multiple categorical variables (CA only does 2)
\item
  Corrects the dimensionality
\item
  Has nearly magical properties (we'll see later)
\end{itemize}

\end{frame}

\begin{frame}

\includegraphics{PCA_MCA_Slides_files/figure-beamer/unnamed-chunk-33-1.pdf}

\end{frame}

\begin{frame}

\includegraphics{PCA_MCA_Slides_files/figure-beamer/unnamed-chunk-34-1.pdf}

\end{frame}

\begin{frame}{New interpretations}
\protect\hypertarget{new-interpretations}{}

\end{frame}

\begin{frame}

\includegraphics{PCA_MCA_Slides_files/figure-beamer/unnamed-chunk-35-1.pdf}

\end{frame}

\begin{frame}

\includegraphics{PCA_MCA_Slides_files/figure-beamer/unnamed-chunk-36-1.pdf}

\end{frame}

\begin{frame}

\includegraphics{PCA_MCA_Slides_files/figure-beamer/unnamed-chunk-37-1.pdf}

\end{frame}

\begin{frame}{Why does it look like that?}
\protect\hypertarget{why-does-it-look-like-that}{}

\end{frame}

\begin{frame}

\begin{table}[H]
\centering
\resizebox{\linewidth}{!}{
\begin{tabular}{rrrrrrr}
\toprule
DX.MCI & DX.CN & DX.Dementia & PTRACCAT.White & PTRACCAT.Other & PTRACCAT.Black & PTRACCAT.Asian\\
\midrule
1 & 0 & 0 & 1 & 0 & 0 & 0\\
1 & 0 & 0 & 0 & 1 & 0 & 0\\
1 & 0 & 0 & 0 & 0 & 1 & 0\\
0 & 1 & 0 & 0 & 1 & 0 & 0\\
1 & 0 & 0 & 0 & 0 & 0 & 1\\
\addlinespace
0 & 1 & 0 & 1 & 0 & 0 & 0\\
0 & 1 & 0 & 0 & 0 & 1 & 0\\
0 & 0 & 1 & 1 & 0 & 0 & 0\\
0 & 1 & 0 & 0 & 0 & 0 & 1\\
0 & 0 & 1 & 0 & 0 & 0 & 1\\
0 & 0 & 1 & 0 & 0 & 1 & 0\\
\bottomrule
\end{tabular}}
\end{table}

These are \emph{all} the possible combinations from all 665

\end{frame}

\begin{frame}{Compare the results}
\protect\hypertarget{compare-the-results}{}

\includegraphics{PCA_MCA_Slides_files/figure-beamer/unnamed-chunk-39-1.pdf}

\end{frame}

\begin{frame}{Compare the results}
\protect\hypertarget{compare-the-results-1}{}

\begin{table}[H]
\centering
\resizebox{\linewidth}{!}{
\begin{tabular}{lrrrrr}
\toprule
  & PCA Comp. 1 & PCA Comp. 2 & PCA Comp. 3 & PCA Comp. 4 & PCA Comp. 5\\
\midrule
MCA Comp. 1 & 0.17 & -0.25 & 0.92 & 0.06 & -0.26\\
MCA Comp. 2 & -0.78 & 0.36 & 0.28 & -0.42 & 0.03\\
\bottomrule
\end{tabular}}
\end{table}

\begin{itemize}[<+->]
\tightlist
\item
  CA \& MCA produce identical results, except MCA:

  \begin{itemize}[<+->]
  \tightlist
  \item
    Drops components
  \item
    Corrects explained variance
  \end{itemize}
\end{itemize}

\end{frame}

\hypertarget{example-1}{%
\subsection{Example}\label{example-1}}

\begin{frame}{Scaling up}
\protect\hypertarget{scaling-up-2}{}

\begin{table}[H]
\centering\begingroup\fontsize{7}{9}\selectfont

\begin{tabular}{lllllrr}
\toprule
  & DX & PTGENDER & PTETHCAT & PTRACCAT & APOE4 & HMSCORE\\
\midrule
5023 & CN & Female & Not Hisp/Latino & Asian & 0 & 0\\
5026 & MCI & Female & Not Hisp/Latino & White & 1 & 1\\
5027 & Dementia & Male & Not Hisp/Latino & White & 0 & 1\\
5028 & Dementia & Male & Not Hisp/Latino & White & 2 & 1\\
5031 & MCI & Female & Hisp/Latino & White & 0 & 1\\
\addlinespace
5037 & Dementia & Male & Not Hisp/Latino & Black & 1 & 1\\
5040 & CN & Female & Not Hisp/Latino & Black & 0 & 1\\
5047 & MCI & Female & Not Hisp/Latino & Black & 2 & 1\\
5054 & Dementia & Female & Not Hisp/Latino & White & 1 & 0\\
5058 & Dementia & Male & Not Hisp/Latino & Asian & 0 & 0\\
5063 & Dementia & Female & Not Hisp/Latino & White & 1 & 1\\
\bottomrule
\end{tabular}\endgroup{}
\end{table}

\end{frame}

\begin{frame}

\includegraphics{PCA_MCA_Slides_files/figure-beamer/unnamed-chunk-43-1.pdf}

\end{frame}

\begin{frame}

\includegraphics{PCA_MCA_Slides_files/figure-beamer/unnamed-chunk-44-1.pdf}

\end{frame}

\begin{frame}

\includegraphics{PCA_MCA_Slides_files/figure-beamer/unnamed-chunk-45-1.pdf}

\end{frame}

\begin{frame}

\includegraphics{PCA_MCA_Slides_files/figure-beamer/unnamed-chunk-46-1.pdf}

\end{frame}

\hypertarget{binomial-data}{%
\subsection{Binomial data}\label{binomial-data}}

\begin{verbatim}
## [1] "Corrections have failed. Original information must be used."
\end{verbatim}

\begin{frame}{A very important detour}
\protect\hypertarget{a-very-important-detour}{}

\begin{table}[H]
\centering\begingroup\fontsize{7}{9}\selectfont

\begin{tabular}{lll}
\toprule
  & PTGENDER & PTETHCAT\\
\midrule
5023 & Female & Not Hisp/Latino\\
5026 & Female & Not Hisp/Latino\\
5027 & Male & Not Hisp/Latino\\
5028 & Male & Not Hisp/Latino\\
5031 & Female & Hisp/Latino\\
\addlinespace
5037 & Male & Not Hisp/Latino\\
5040 & Female & Not Hisp/Latino\\
5047 & Female & Not Hisp/Latino\\
5054 & Female & Not Hisp/Latino\\
5058 & Male & Not Hisp/Latino\\
5063 & Female & Not Hisp/Latino\\
\bottomrule
\end{tabular}\endgroup{}
\end{table}

\begin{center}Two variables with strictly two levels (i.e., binary data)\end{center}

\end{frame}

\begin{frame}{A very important detour}
\protect\hypertarget{a-very-important-detour-1}{}

\begin{table}[H]
\centering
\resizebox{\linewidth}{!}{
\begin{tabular}{lrrrr}
\toprule
  & PTGENDER.Male & PTGENDER.Female & PTETHCAT.Not Hisp/Latino & PTETHCAT.Hisp/Latino\\
\midrule
5023 & 0 & 1 & 1 & 0\\
5026 & 0 & 1 & 1 & 0\\
5027 & 1 & 0 & 1 & 0\\
5028 & 1 & 0 & 1 & 0\\
5031 & 0 & 1 & 0 & 1\\
\addlinespace
5037 & 1 & 0 & 1 & 0\\
5040 & 0 & 1 & 1 & 0\\
5047 & 0 & 1 & 1 & 0\\
5054 & 0 & 1 & 1 & 0\\
5058 & 1 & 0 & 1 & 0\\
5063 & 0 & 1 & 1 & 0\\
\bottomrule
\end{tabular}}
\end{table}

\begin{center}Disjunctive coding of two variables with strictly two levels (i.e., binary data) into four columns\end{center}

\end{frame}

\begin{frame}{A very important detour}
\protect\hypertarget{a-very-important-detour-2}{}

\begin{table}[H]
\centering\begingroup\fontsize{7}{9}\selectfont

\begin{tabular}{lll}
\toprule
  & PTGENDER & PTETHCAT\\
\midrule
5023 & Female & Not Hisp/Latino\\
5026 & Female & Not Hisp/Latino\\
5027 & Male & Not Hisp/Latino\\
5028 & Male & Not Hisp/Latino\\
5031 & Female & Hisp/Latino\\
\addlinespace
5037 & Male & Not Hisp/Latino\\
5040 & Female & Not Hisp/Latino\\
5047 & Female & Not Hisp/Latino\\
5054 & Female & Not Hisp/Latino\\
5058 & Male & Not Hisp/Latino\\
5063 & Female & Not Hisp/Latino\\
\bottomrule
\end{tabular}\endgroup{}
\end{table}

\begin{center}Two variables with strictly two levels (i.e., binary data)\end{center}

\end{frame}

\begin{frame}{A very important detour}
\protect\hypertarget{a-very-important-detour-3}{}

\begin{table}[H]
\centering\begingroup\fontsize{7}{9}\selectfont

\begin{tabular}{lrr}
\toprule
  & PTGENDER & PTETHCAT\\
\midrule
5023 & 1 & 0\\
5026 & 1 & 0\\
5027 & 0 & 0\\
5028 & 0 & 0\\
5031 & 1 & 1\\
\addlinespace
5037 & 0 & 0\\
5040 & 1 & 0\\
5047 & 1 & 0\\
5054 & 1 & 0\\
5058 & 0 & 0\\
5063 & 1 & 0\\
\bottomrule
\end{tabular}\endgroup{}
\end{table}

\begin{center}Binary coding of two variables with strictly two levels (i.e., binary data) in two columns\end{center}

\end{frame}

\begin{frame}{A very important detour}
\protect\hypertarget{a-very-important-detour-4}{}

\begin{table}[H]
\centering\begingroup\fontsize{7}{9}\selectfont

\begin{tabular}{lrr}
\toprule
  & PTGENDER & PTETHCAT\\
\midrule
5023 & 0 & 1\\
5026 & 0 & 1\\
5027 & 1 & 1\\
5028 & 1 & 1\\
5031 & 0 & 0\\
\addlinespace
5037 & 1 & 1\\
5040 & 0 & 1\\
5047 & 0 & 1\\
5054 & 0 & 1\\
5058 & 1 & 1\\
5063 & 0 & 1\\
\bottomrule
\end{tabular}\endgroup{}
\end{table}

\begin{center}Alternate but equivalent binary coding of two variables with strictly two levels (i.e., binary data) in two columns\end{center}

\end{frame}

\begin{frame}{Always a bad idea?}
\protect\hypertarget{always-a-bad-idea}{}

\begin{itemize}[<+->]
\tightlist
\item
  MCA on the disjunctive coded data
\item
  PCA on the binary coded data
\end{itemize}

\end{frame}

\begin{frame}

\includegraphics{PCA_MCA_Slides_files/figure-beamer/unnamed-chunk-53-1.pdf}

\end{frame}

\begin{frame}

\includegraphics{PCA_MCA_Slides_files/figure-beamer/unnamed-chunk-54-1.pdf}

\begin{center}Oh, weird!\end{center}

\end{frame}

\begin{frame}

\includegraphics{PCA_MCA_Slides_files/figure-beamer/unnamed-chunk-55-1.pdf}

\begin{center}Component 2 is "flipped" \\
We will revisit this
\end{center}

\end{frame}

\begin{frame}

\begin{table}[H]
\centering
\resizebox{\linewidth}{!}{
\begin{tabular}{lrr}
\toprule
  & PCA Comp. 1 & PCA Comp. 2\\
\midrule
MCA Comp. 1 & 1 & 0\\
MCA Comp. 2 & 0 & -1\\
\bottomrule
\end{tabular}}
\end{table}
\begin{center}Oh, double weird!\end{center}

\end{frame}

\begin{frame}{Let's get weird}
\protect\hypertarget{lets-get-weird}{}

\begin{table}[H]
\centering
\resizebox{\linewidth}{!}{
\begin{tabular}{lrr}
\toprule
  & PTGENDER & PTETHCAT\\
\midrule
PTGENDER & 1.00 & 0.06\\
PTETHCAT & 0.06 & 1.00\\
\bottomrule
\end{tabular}}
\end{table}

\end{frame}

\begin{frame}{Let's get weird}
\protect\hypertarget{lets-get-weird-1}{}

\begin{table}[H]
\centering
\resizebox{\linewidth}{!}{
\begin{tabular}{lrr}
\toprule
  & PTGENDER & PTETHCAT\\
\midrule
PTGENDER & 1.00 & 0.06\\
PTETHCAT & 0.06 & 1.00\\
\bottomrule
\end{tabular}}
\end{table}

\begin{itemize}[<+->]
\item
  \(\phi=\) 0.06
\item
  Deep connections between \(\chi^2\), Normal, binomial (and others)
\item
  We can expand the idea of ``binary'' or ``binomial''
\end{itemize}

\end{frame}

\hypertarget{continuous-data}{%
\subsection{Continuous data}\label{continuous-data}}

\begin{frame}{An old friend}
\protect\hypertarget{an-old-friend}{}

\begin{table}[H]
\centering\begingroup\fontsize{7}{9}\selectfont

\begin{tabular}{lrr}
\toprule
  & mPACCtrailsB & FDG\\
\midrule
5023 & 1.12 & 0.13\\
5026 & 0.46 & -1.31\\
5027 & -2.77 & -1.48\\
5028 & -1.59 & -0.97\\
5031 & -0.92 & -0.87\\
\addlinespace
5037 & -1.86 & -2.00\\
5040 & 0.94 & -0.21\\
5047 & -0.25 & 3.05\\
5054 & -0.80 & -1.05\\
5058 & -1.12 & -2.13\\
5063 & -2.31 & -2.49\\
\bottomrule
\end{tabular}\endgroup{}
\end{table}
\begin{center}We perform(ed) PCA on these data\end{center}

\end{frame}

\begin{frame}{Escofier's Geometric Magic}
\protect\hypertarget{escofiers-geometric-magic}{}

\begin{itemize}[<+->]
\tightlist
\item
  One of the ``fuzzy'' or ``bipolar'' coding schemes
\item
  Take each Z-scored continuous variable
\item
  Duplicate it as
  \(\begin{bmatrix} \frac{1-Z}{2} \frac{1+Z}{2}\end{bmatrix}\)
\end{itemize}

\end{frame}

\begin{frame}{Escofier's Geometric Magic}
\protect\hypertarget{escofiers-geometric-magic-1}{}

\begin{table}[H]
\centering\begingroup\fontsize{7}{9}\selectfont

\begin{tabular}{lrrrr}
\toprule
  & mPACCtrailsB- & mPACCtrailsB+ & FDG- & FDG+\\
\midrule
5023 & -0.06 & 1.06 & 0.43 & 0.57\\
5026 & 0.27 & 0.73 & 1.16 & -0.16\\
5027 & 1.88 & -0.88 & 1.24 & -0.24\\
5028 & 1.30 & -0.30 & 0.98 & 0.02\\
5031 & 0.96 & 0.04 & 0.93 & 0.07\\
\addlinespace
5037 & 1.43 & -0.43 & 1.50 & -0.50\\
5040 & 0.03 & 0.97 & 0.60 & 0.40\\
5047 & 0.62 & 0.38 & -1.03 & 2.03\\
5054 & 0.90 & 0.10 & 1.03 & -0.03\\
5058 & 1.06 & -0.06 & 1.57 & -0.57\\
5063 & 1.66 & -0.66 & 1.74 & -0.74\\
\bottomrule
\end{tabular}\endgroup{}
\end{table}

\end{frame}

\begin{frame}{Escofier's Geometric Magic}
\protect\hypertarget{escofiers-geometric-magic-2}{}

\begin{table}[H]
\centering\begingroup\fontsize{7}{9}\selectfont

\begin{tabular}{lrrrr}
\toprule
  & mPACCtrailsB- & mPACCtrailsB+ & FDG- & FDG+\\
\midrule
5023 & -0.06 & 1.06 & 0.43 & 0.57\\
5026 & 0.27 & 0.73 & 1.16 & -0.16\\
5027 & 1.88 & -0.88 & 1.24 & -0.24\\
5028 & 1.30 & -0.30 & 0.98 & 0.02\\
5031 & 0.96 & 0.04 & 0.93 & 0.07\\
\addlinespace
5037 & 1.43 & -0.43 & 1.50 & -0.50\\
5040 & 0.03 & 0.97 & 0.60 & 0.40\\
5047 & 0.62 & 0.38 & -1.03 & 2.03\\
5054 & 0.90 & 0.10 & 1.03 & -0.03\\
5058 & 1.06 & -0.06 & 1.57 & -0.57\\
5063 & 1.66 & -0.66 & 1.74 & -0.74\\
\bottomrule
\end{tabular}\endgroup{}
\end{table}

\begin{itemize}[<+->]
\tightlist
\item
  Row sums are total number of \emph{original} variables
\item
  Sum within a variable (e.g., FDG) is total number of rows
\item
  Sum of the table is rows \(\times\) columns
\item
  \emph{These behave like disjunctive data!}
\end{itemize}

\end{frame}

\begin{frame}

\includegraphics{PCA_MCA_Slides_files/figure-beamer/unnamed-chunk-64-1.pdf}

\begin{center}Oh, interesting!\\
Take note: each variable has two "poles"\end{center}

\end{frame}

\begin{frame}

\includegraphics{PCA_MCA_Slides_files/figure-beamer/unnamed-chunk-65-1.pdf}

\begin{center}Oh, weird!\end{center}

\end{frame}

\begin{frame}

\includegraphics{PCA_MCA_Slides_files/figure-beamer/unnamed-chunk-66-1.pdf}

\begin{center}Oh, double weird!\end{center}

\end{frame}

\begin{frame}

\includegraphics{PCA_MCA_Slides_files/figure-beamer/unnamed-chunk-67-1.pdf}

\begin{center}Flips: They don't matter.\end{center}

\end{frame}

\begin{frame}

\begin{table}[H]
\centering
\resizebox{\linewidth}{!}{
\begin{tabular}{lrr}
\toprule
  & PCA Comp. 1 & PCA Comp. 2\\
\midrule
CA Comp. 1 & 1 & 0\\
CA Comp. 2 & 0 & -1\\
\bottomrule
\end{tabular}}
\end{table}
\begin{center}Flips: They don't matter.\end{center}

\end{frame}

\begin{frame}{Escofier's Geometric Trick}
\protect\hypertarget{escofiers-geometric-trick}{}

\begin{itemize}[<+->]
\tightlist
\item
  Apply PCA to continuous data or
\item
  Apply CA to ``Escofier transformed'' data
\end{itemize}

\end{frame}

\hypertarget{ordinal-data}{%
\subsection{Ordinal data}\label{ordinal-data}}

\begin{frame}{Thermometer}
\protect\hypertarget{thermometer}{}

\begin{itemize}[<+->]
\tightlist
\item
  For ordinal data
\item
  Another ``fuzzy'' or ``bipolar'' coding
\item
  More Escofier Geometric Magic

  \begin{itemize}[<+->]
  \tightlist
  \item
    Subtract the maximum (minimum is now \(0\))
  \item
    \(\begin{bmatrix} \frac{\texttt{max}(x)-x}{\texttt{max}} \frac{x-\texttt{min}(x)}{\texttt{max}}\end{bmatrix}\)
  \end{itemize}
\item
  Apply CA
\end{itemize}

\end{frame}

\begin{frame}{More Geometric Magic}
\protect\hypertarget{more-geometric-magic}{}

\begin{table}[H]
\centering\begingroup\fontsize{7}{9}\selectfont

\begin{tabular}{lrrrr}
\toprule
  & PTEDUCAT & CDRSB & ADAS13 & MOCA\\
\midrule
5023 & 18 & 0.0 & 6 & 30\\
5026 & 18 & 1.5 & 8 & 24\\
5027 & 18 & 4.0 & 27 & 19\\
5028 & 16 & 3.5 & 20 & 19\\
5031 & 14 & 2.0 & 16 & 20\\
\addlinespace
5037 & 16 & 5.0 & 35 & 17\\
5040 & 18 & 0.0 & 8 & 20\\
5047 & 16 & 1.0 & 17 & 24\\
5054 & 18 & 3.5 & 22 & 21\\
5058 & 20 & 3.0 & 17 & 21\\
5063 & 14 & 2.5 & 38 & 16\\
\bottomrule
\end{tabular}\endgroup{}
\end{table}

\end{frame}

\begin{frame}{More Geometric Magic}
\protect\hypertarget{more-geometric-magic-1}{}

\begin{table}[H]
\centering
\resizebox{\linewidth}{!}{
\begin{tabular}{lrrrrrrrr}
\toprule
  & PTEDUCAT+ & PTEDUCAT- & CDRSB+ & CDRSB- & ADAS13+ & ADAS13- & MOCA+ & MOCA-\\
\midrule
5023 & 0.75 & 0.25 & 0.00 & 1.00 & 0.13 & 0.87 & 1.00 & 0.00\\
5026 & 0.75 & 0.25 & 0.27 & 0.73 & 0.17 & 0.83 & 0.57 & 0.43\\
5027 & 0.75 & 0.25 & 0.73 & 0.27 & 0.59 & 0.41 & 0.21 & 0.79\\
5028 & 0.50 & 0.50 & 0.64 & 0.36 & 0.43 & 0.57 & 0.21 & 0.79\\
5031 & 0.25 & 0.75 & 0.36 & 0.64 & 0.35 & 0.65 & 0.29 & 0.71\\
\addlinespace
5037 & 0.50 & 0.50 & 0.91 & 0.09 & 0.76 & 0.24 & 0.07 & 0.93\\
5040 & 0.75 & 0.25 & 0.00 & 1.00 & 0.17 & 0.83 & 0.29 & 0.71\\
5047 & 0.50 & 0.50 & 0.18 & 0.82 & 0.37 & 0.63 & 0.57 & 0.43\\
5054 & 0.75 & 0.25 & 0.64 & 0.36 & 0.48 & 0.52 & 0.36 & 0.64\\
5058 & 1.00 & 0.00 & 0.55 & 0.45 & 0.37 & 0.63 & 0.36 & 0.64\\
5063 & 0.25 & 0.75 & 0.45 & 0.55 & 0.83 & 0.17 & 0.00 & 1.00\\
\bottomrule
\end{tabular}}
\end{table}

\end{frame}

\begin{frame}{More Geometric Magic}
\protect\hypertarget{more-geometric-magic-2}{}

\begin{table}[H]
\centering
\resizebox{\linewidth}{!}{
\begin{tabular}{lrrrrrrrr}
\toprule
  & PTEDUCAT+ & PTEDUCAT- & CDRSB+ & CDRSB- & ADAS13+ & ADAS13- & MOCA+ & MOCA-\\
\midrule
5023 & 0.75 & 0.25 & 0.00 & 1.00 & 0.13 & 0.87 & 1.00 & 0.00\\
5026 & 0.75 & 0.25 & 0.27 & 0.73 & 0.17 & 0.83 & 0.57 & 0.43\\
5027 & 0.75 & 0.25 & 0.73 & 0.27 & 0.59 & 0.41 & 0.21 & 0.79\\
5028 & 0.50 & 0.50 & 0.64 & 0.36 & 0.43 & 0.57 & 0.21 & 0.79\\
5031 & 0.25 & 0.75 & 0.36 & 0.64 & 0.35 & 0.65 & 0.29 & 0.71\\
\addlinespace
5037 & 0.50 & 0.50 & 0.91 & 0.09 & 0.76 & 0.24 & 0.07 & 0.93\\
5040 & 0.75 & 0.25 & 0.00 & 1.00 & 0.17 & 0.83 & 0.29 & 0.71\\
5047 & 0.50 & 0.50 & 0.18 & 0.82 & 0.37 & 0.63 & 0.57 & 0.43\\
5054 & 0.75 & 0.25 & 0.64 & 0.36 & 0.48 & 0.52 & 0.36 & 0.64\\
5058 & 1.00 & 0.00 & 0.55 & 0.45 & 0.37 & 0.63 & 0.36 & 0.64\\
5063 & 0.25 & 0.75 & 0.45 & 0.55 & 0.83 & 0.17 & 0.00 & 1.00\\
\bottomrule
\end{tabular}}
\end{table}

\begin{itemize}[<+->]
\tightlist
\item
  Row sums are total number of \emph{original} variables
\item
  Sum within a variable (e.g.~EDU) is total number of rows
\item
  Sum of the table is rows \(\times\) columns
\item
  \emph{These behave like disjunctive data!}
\end{itemize}

\end{frame}

\begin{frame}{Let's take a look}
\protect\hypertarget{lets-take-a-look}{}

\end{frame}

\begin{frame}

\includegraphics{PCA_MCA_Slides_files/figure-beamer/unnamed-chunk-73-1.pdf}

\end{frame}

\begin{frame}

\includegraphics{PCA_MCA_Slides_files/figure-beamer/unnamed-chunk-74-1.pdf}

\end{frame}

\begin{frame}

\includegraphics{PCA_MCA_Slides_files/figure-beamer/unnamed-chunk-75-1.pdf}
Special properties: Biploar coding passes through 0.

\end{frame}

\begin{frame}

\includegraphics{PCA_MCA_Slides_files/figure-beamer/unnamed-chunk-76-1.pdf}

\end{frame}

\begin{frame}

\includegraphics{PCA_MCA_Slides_files/figure-beamer/unnamed-chunk-77-1.pdf}

\end{frame}

\begin{frame}

\includegraphics{PCA_MCA_Slides_files/figure-beamer/unnamed-chunk-78-1.pdf}

\end{frame}

\hypertarget{ordinal-vs.-disjunctive}{%
\subsection{Ordinal vs.~Disjunctive}\label{ordinal-vs.-disjunctive}}

\begin{frame}{Thermometer vs.~Disjunctive}
\protect\hypertarget{thermometer-vs.-disjunctive}{}

\begin{itemize}[<+->]
\tightlist
\item
  Sometimes data could be either
\item
  Let's analyze it both ways
\end{itemize}

\end{frame}

\begin{frame}

\begin{table}[H]
\centering\begingroup\fontsize{7}{9}\selectfont

\begin{tabular}{lrr}
\toprule
  & APOE4 & HMSCORE\\
\midrule
5023 & 0 & 0\\
5026 & 1 & 1\\
5027 & 0 & 1\\
5028 & 2 & 1\\
5031 & 0 & 1\\
\addlinespace
5037 & 1 & 1\\
5040 & 0 & 1\\
5047 & 2 & 1\\
5054 & 1 & 0\\
5058 & 0 & 0\\
5063 & 1 & 1\\
\bottomrule
\end{tabular}\endgroup{}
\end{table}

\end{frame}

\begin{frame}

\includegraphics{PCA_MCA_Slides_files/figure-beamer/unnamed-chunk-81-1.pdf}

\end{frame}

\begin{frame}

\includegraphics{PCA_MCA_Slides_files/figure-beamer/unnamed-chunk-82-1.pdf}

\end{frame}

\begin{frame}{Thermometer vs.~Disjunctive}
\protect\hypertarget{thermometer-vs.-disjunctive-1}{}

\begin{itemize}[<+->]
\tightlist
\item
  For a small (reasonable) number of levels: disjunctive
\item
  Otherwise: thermometer
\item
  Interpretation:

  \begin{itemize}[<+->]
  \tightlist
  \item
    Thermometer is ``easier''
  \item
    Disjunctive is more informative
  \end{itemize}
\end{itemize}

\end{frame}

\hypertarget{mixed-data}{%
\subsection{Mixed data}\label{mixed-data}}

\begin{frame}{All of the data}
\protect\hypertarget{all-of-the-data}{}

\end{frame}

\begin{frame}{All of the data}
\protect\hypertarget{all-of-the-data-1}{}

\begin{table}[H]
\centering
\resizebox{\linewidth}{!}{
\begin{tabular}{llrlrllrrrrrrrrrrr}
\toprule
  & DX & AGE & PTGENDER & PTEDUCAT & PTETHCAT & PTRACCAT & APOE4 & FDG & AV45 & CDRSB & ADAS13 & MOCA & WholeBrain & Hippocampus & MidTemp & mPACCtrailsB & HMSCORE\\
\midrule
5023 & CN & 63.9 & Female & 18 & Not Hisp/Latino & Asian & 0 & 1.29 & 1.03 & 0.0 & 6 & 30 & 1057351.0 & 7904 & 21306 & 1.81 & 0\\
5026 & MCI & 70.5 & Female & 18 & Not Hisp/Latino & White & 1 & 1.08 & 1.44 & 1.5 & 8 & 24 & 1023057.3 & 8051 & 16501 & -1.45 & 1\\
5027 & Dementia & 75.5 & Male & 18 & Not Hisp/Latino & White & 0 & 1.06 & 1.44 & 4.0 & 27 & 19 & 986723.7 & 6534 & 17437 & -17.27 & 1\\
5028 & Dementia & 61.9 & Male & 16 & Not Hisp/Latino & White & 2 & 1.13 & 1.38 & 3.5 & 20 & 19 & 1182704.6 & 7481 & 20797 & -11.50 & 1\\
5031 & MCI & 80.2 & Female & 14 & Hisp/Latino & White & 0 & 1.14 & 1.52 & 2.0 & 16 & 20 & 908133.9 & 5040 & 19032 & -8.21 & 1\\
\addlinespace
5037 & Dementia & 67.3 & Male & 16 & Not Hisp/Latino & Black & 1 & 0.98 & 1.21 & 5.0 & 35 & 17 & 1161499.6 & 5831 & 21428 & -12.80 & 1\\
5040 & CN & 75.9 & Female & 18 & Not Hisp/Latino & Black & 0 & 1.24 & 1.01 & 0.0 & 8 & 20 & 943160.6 & 7994 & 16634 & 0.94 & 1\\
5047 & MCI & 68.8 & Female & 16 & Not Hisp/Latino & Black & 2 & 1.70 & 1.48 & 1.0 & 17 & 24 & 1070406.1 & 7920 & 22043 & -4.90 & 1\\
5054 & Dementia & 74.0 & Female & 18 & Not Hisp/Latino & White & 1 & 1.12 & 1.43 & 3.5 & 22 & 21 & 1138040.1 & 6580 & 20836 & -7.63 & 0\\
5058 & Dementia & 61.8 & Male & 20 & Not Hisp/Latino & Asian & 0 & 0.97 & 1.54 & 3.0 & 17 & 21 & 1195549.3 & 7318 & 22757 & -9.18 & 0\\
5063 & Dementia & 71.5 & Female & 14 & Not Hisp/Latino & White & 1 & 0.92 & 1.61 & 2.5 & 38 & 16 & 817421.2 & 5364 & 12542 & -15.03 & 1\\
\bottomrule
\end{tabular}}
\end{table}

\end{frame}

\begin{frame}

\includegraphics{PCA_MCA_Slides_files/figure-beamer/unnamed-chunk-85-1.pdf}

\end{frame}

\begin{frame}

\includegraphics{PCA_MCA_Slides_files/figure-beamer/unnamed-chunk-86-1.pdf}

\end{frame}

\hypertarget{resampling}{%
\section{Resampling}\label{resampling}}

\hypertarget{overview-1}{%
\subsection{Overview}\label{overview-1}}

\begin{frame}{Background}
\protect\hypertarget{background}{}

\begin{itemize}[<+->]
\tightlist
\item
  Generally the Gifi or Benzecri principles
\item
  Benzecri

  \begin{itemize}[<+->]
  \tightlist
  \item
    ``statistics was built a pompous discipline based on theoretical
    assumptions that are rarely met in practice''
  \item
    ``the models should follow the data, not vice versa''
  \item
    ``use the computer implies the abandonment of all the techniques
    designed before of computing''
  \end{itemize}
\item
  Gifi

  \begin{itemize}[<+->]
  \tightlist
  \item
    Replication stability: new data, same techniques
  \item
    Selection stability: Data variations
  \item
    Technique stability: Different technique, same data
  \end{itemize}
\item
  Pause!
\end{itemize}

\end{frame}

\begin{frame}{My beliefs}
\protect\hypertarget{my-beliefs}{}

\begin{itemize}[<+->]
\tightlist
\item
  Might give you inference/generalizability

  \begin{itemize}[<+->]
  \tightlist
  \item
    Depending on data, design, etc\ldots{}
  \end{itemize}
\item
  Practically

  \begin{itemize}[<+->]
  \tightlist
  \item
    Assessing stability of \emph{your} data
  \item
    Provides critical diagnostics
  \end{itemize}
\end{itemize}

\end{frame}

\begin{frame}{Definitions}
\protect\hypertarget{definitions-1}{}

\begin{itemize}[<+->]
\tightlist
\item
  Permutation: break relationships in the data
\item
  Split-half: mutually exclusive sets
\item
  Bootstrap: resample with reselection
\end{itemize}

\end{frame}

\begin{frame}

\begin{figure}
\centering
\includegraphics{../Images/base_data.png}
\caption{Tiny illustrative data}
\end{figure}

\end{frame}

\begin{frame}

\begin{figure}
\centering
\includegraphics{../Images/permutation_data.png}
\caption{Tiny permuted illustrative data}
\end{figure}

\end{frame}

\begin{frame}

\begin{figure}
\centering
\includegraphics{../Images/split_half_data.png}
\caption{Tiny split half illustrative data}
\end{figure}

\end{frame}

\begin{frame}

\begin{figure}
\centering
\includegraphics{../Images/bootstrap_data.png}
\caption{Tiny bootstrap illustrative data}
\end{figure}

\end{frame}

\begin{frame}{Uses in PCA \& CA}
\protect\hypertarget{uses-in-pca-ca}{}

\begin{itemize}[<+->]
\tightlist
\item
  Permutation: Effect size tests of components
\item
  Split-half: Replication of components
\item
  Bootstrap: Stability of variables
\end{itemize}

\end{frame}

\hypertarget{permutation}{%
\subsection{Permutation}\label{permutation}}

\begin{frame}

\begin{figure}
\centering
\includegraphics{../Images/permutation_diagram.png}
\caption{Permutation diagram}
\end{figure}

\end{frame}

\begin{frame}

\begin{figure}
\centering
\includegraphics{../Images/Perm1.png}
\caption{Permutation: First component}
\end{figure}

\end{frame}

\begin{frame}

\begin{figure}
\centering
\includegraphics{../Images/Perm2.png}
\caption{Permutation: Second component}
\end{figure}

\end{frame}

\begin{frame}

\begin{figure}
\centering
\includegraphics{../Images/Perm3.png}
\caption{Permutation: Third component}
\end{figure}

\end{frame}

\begin{frame}

\begin{figure}
\centering
\includegraphics{../Images/Perm4.png}
\caption{Permutation: Fifth component}
\end{figure}

\end{frame}

\begin{frame}

\begin{figure}
\centering
\includegraphics{../Images/Perm5.png}
\caption{Permutation: p-values}
\end{figure}

\end{frame}

\begin{frame}{Another detour}
\protect\hypertarget{another-detour}{}

\begin{itemize}[<+->]
\tightlist
\item
  p-values should (inversely) follow the scree
\item
  Diagnostic tests:

  \begin{itemize}[<+->]
  \tightlist
  \item
    Large or erratic jumps
  \item
    First or first few \(ps \geq .5\)
  \end{itemize}
\end{itemize}

\end{frame}

\begin{frame}{Conclusions}
\protect\hypertarget{conclusions}{}

\begin{itemize}[<+->]
\tightlist
\item
  First few components have larger than expected effect sizes

  \begin{itemize}[<+->]
  \tightlist
  \item
    More variance than null
  \end{itemize}
\item
  We do not know if these generalize
\end{itemize}

\end{frame}

\hypertarget{split-half}{%
\subsection{Split-half}\label{split-half}}

\begin{frame}

\begin{figure}
\centering
\includegraphics{../Images/split_half_diagram.png}
\caption{Split half diagram}
\end{figure}

\end{frame}

\begin{frame}

\begin{figure}
\centering
\includegraphics{../Images/split1.png}
\caption{Split-half correlations: First component}
\end{figure}

\end{frame}

\begin{frame}

\begin{figure}
\centering
\includegraphics{../Images/split2.png}
\caption{Split-half correlations: Second component}
\end{figure}

\end{frame}

\begin{frame}

\begin{figure}
\centering
\includegraphics{../Images/split3.png}
\caption{Split-half correlations: Third component}
\end{figure}

\end{frame}

\begin{frame}

\begin{figure}
\centering
\includegraphics{../Images/split4.png}
\caption{Split-half correlations: All components}
\end{figure}

\end{frame}

\begin{frame}{Conclusions}
\protect\hypertarget{conclusions-1}{}

\begin{itemize}[<+->]
\tightlist
\item
  We do \emph{sort of} know if these generalize
\item
  First two really do
\item
  Next few: Maybe
\item
  Key observation:

  \begin{itemize}[<+->]
  \tightlist
  \item
    Components \emph{flip} order!
  \item
    We need to question the meaning of order of components in our data
  \end{itemize}
\end{itemize}

\end{frame}

\hypertarget{bootstrap}{%
\subsection{Bootstrap}\label{bootstrap}}

\begin{frame}{Bootstrap goes last}
\protect\hypertarget{bootstrap-goes-last}{}

\begin{itemize}[<+->]
\tightlist
\item
  You need to know the number of components to interpret
\item
  We have 2
\item
  Now you can interpret variables \emph{per} component

  \begin{itemize}[<+->]
  \tightlist
  \item
    Find the ones that are stable
  \end{itemize}
\end{itemize}

\end{frame}

\begin{frame}

\begin{figure}
\centering
\includegraphics{../Images/bootstrap_diagram.png}
\caption{Bootstrap diagram}
\end{figure}

\end{frame}

\begin{frame}

\begin{figure}
\centering
\includegraphics{../Images/boot1.png}
\caption{Bootstrap ratios}
\end{figure}

\end{frame}

\begin{frame}

\begin{figure}
\centering
\includegraphics{../Images/boot2.png}
\caption{Bootstrap ratios}
\end{figure}

\end{frame}

\begin{frame}

\begin{figure}
\centering
\includegraphics{../Images/boot3.png}
\caption{Bootstrap ratios}
\end{figure}

\end{frame}

\begin{frame}

\begin{figure}
\centering
\includegraphics{../Images/boot4.png}
\caption{Bootstrap ratios}
\end{figure}

\end{frame}

\begin{frame}

\begin{figure}
\centering
\includegraphics{../Images/boot5.png}
\caption{Bootstrap ratios}
\end{figure}

\end{frame}

\begin{frame}

\begin{figure}
\centering
\includegraphics{../Images/boot6.png}
\caption{Bootstrap ratios}
\end{figure}

\end{frame}

\begin{frame}{Conclusions}
\protect\hypertarget{conclusions-2}{}

\begin{itemize}[<+->]
\tightlist
\item
  Just a snapshot (there are more variables)
\item
  APOE4 contributes to both
\item
  The others generally contribute to one or the other
\end{itemize}

\end{frame}

\hypertarget{final-stretch}{%
\section{Final stretch}\label{final-stretch}}

\begin{frame}{Things not discussed}
\protect\hypertarget{things-not-discussed}{}

\begin{itemize}[<+->]
\tightlist
\item
  Corrections
\item
  Alternate preprocessing

  \begin{itemize}[<+->]
  \tightlist
  \item
    There's a lazy way (rank)
  \end{itemize}
\item
  Other resampling \& Cross-validation loops.

  \begin{itemize}[<+->]
  \tightlist
  \item
    Start at the ``beginning''
  \end{itemize}
\item
  What about other data types?

  \begin{itemize}[<+->]
  \tightlist
  \item
    I've actually misled you a bit
  \item
    Structural data
  \end{itemize}
\item
  Rotations

  \begin{itemize}[<+->]
  \tightlist
  \item
    I don't rotate
  \item
    But I won't stop you from it
  \item
    Report both
  \end{itemize}
\end{itemize}

\end{frame}

\begin{frame}{Rotation}
\protect\hypertarget{rotation}{}

\begin{itemize}[<+->]
\tightlist
\item
  Two compelling examples rotation

  \begin{itemize}[<+->]
  \tightlist
  \item
    That weren't rotated
  \item
    Why?
  \end{itemize}
\item
  CA of Mueller report

  \begin{itemize}[<+->]
  \tightlist
  \item
    see \url{http://github.com/derekbeaton/muellerreport_ca}
  \end{itemize}
\item
  CA of NeuroSynth (Alhazmi et al., 2018)

  \begin{itemize}[<+->]
  \tightlist
  \item
    see \url{http://github.com/fahd09/neurosynth_semantic_map}
  \end{itemize}
\end{itemize}

\end{frame}

\begin{frame}

\includegraphics{../Images/mueller_report_CA.png}

\end{frame}

\begin{frame}

\includegraphics{../Images/FA_HBM.jpg}

\end{frame}

\hypertarget{related-techniques}{%
\subsection{Related techniques}\label{related-techniques}}

\begin{frame}{One table}
\protect\hypertarget{one-table}{}

\begin{itemize}[<+->]
\tightlist
\item
  Independent Components Analysis

  \begin{itemize}[<+->]
  \tightlist
  \item
    Effectively a rotation
  \end{itemize}
\item
  Factor analyses, mostly

  \begin{itemize}[<+->]
  \tightlist
  \item
    Different error terms + rotation
  \end{itemize}
\item
  Non-negative matrix factorization
\item
  Non-symmetric CA
\item
  Hellinger CA
\item
  Compositional CA
\item
  Multidimensional scaling (MDS)
\end{itemize}

\end{frame}

\begin{frame}{Two tables: Part 1}
\protect\hypertarget{two-tables-part-1}{}

\begin{itemize}[<+->]
\tightlist
\item
  Partial least squares (correlation)
\item
  Partial least squares (regression)
\item
  Partial least squares (path modelling)
\end{itemize}

\end{frame}

\begin{frame}{Two tables: Part 2}
\protect\hypertarget{two-tables-part-2}{}

\begin{itemize}[<+->]
\tightlist
\item
  Canonical Correlation Analysis
\item
  Discriminant analyses
\item
  Reduced rank regression/redundancy analysis
\item
  Generalized PLS regression

  \begin{itemize}[<+->]
  \tightlist
  \item
    Beaton, Saporta, Abdi (2019)
  \item
    Mixed data, most two table techniques
  \end{itemize}
\end{itemize}

\end{frame}

\begin{frame}{More than two tables}
\protect\hypertarget{more-than-two-tables}{}

\begin{itemize}[<+->]
\tightlist
\item
  STATIS
\item
  Multiple factor analysis
\end{itemize}

\end{frame}

\begin{frame}{Not as related}
\protect\hypertarget{not-as-related}{}

\begin{itemize}[<+->]
\tightlist
\item
  tSNE
\item
  UMAP
\item
  More akin to non-metric multidimensional scaling

  \begin{itemize}[<+->]
  \tightlist
  \item
    Not always a fair comparison
  \end{itemize}
\end{itemize}

\end{frame}

\begin{frame}{For all types of data}
\protect\hypertarget{for-all-types-of-data}{}

\begin{itemize}[<+->]
\tightlist
\item
  Distances (MDS, DiSTATIS, CovSTATIS)
\item
  Networks (CA)

  \begin{itemize}[<+->]
  \tightlist
  \item
    More magic!
  \end{itemize}
\end{itemize}

\end{frame}

\hypertarget{some-references}{%
\section{(Some) References}\label{some-references}}

\begin{frame}{See the reference sections of these}
\protect\hypertarget{see-the-reference-sections-of-these}{}

\begin{itemize}[<+->]
\item
  Beaton, D., Saporta, G., Abdi, H., \& Alzheimer's Disease Neuroimaging
  Initiative. (2019). A generalization of partial least squares
  regression and correspondence analysis for categorical and mixed data:
  An application with the ADNI data. bioRxiv, 598888.
\item
  Beaton, D., Sunderland, K. M., Levine, B., Mandzia, J., Masellis, M.,
  Swartz, R. H., \ldots{} \& Strother, S. C. (2019). Generalization of
  the minimum covariance determinant algorithm for categorical and mixed
  data types. bioRxiv, 333005.
\end{itemize}

\end{frame}

\begin{frame}{And these}
\protect\hypertarget{and-these}{}

\begin{itemize}[<+->]
\item
  Abdi, H., Guillemot, V., Eslami, A., \& Beaton, D. (2017). Canonical
  correlation analysis. Encyclopedia of Social Network Analysis and
  Mining, 1-16.
\item
  Beaton, D., Dunlop, J., \& Abdi, H. (2016). Partial least squares
  correspondence analysis: A framework to simultaneously analyze
  behavioral and genetic data. Psychological methods, 21(4), 621.
\end{itemize}

\end{frame}

\begin{frame}{Techniques}
\protect\hypertarget{techniques}{}

\begin{itemize}[<+->]
\item
  Greenacre, M. (2017). Correspondence analysis in practice. CRC press.
\item
  Greenacre, M. J. (1984). Theory and Applications of Correspondence
  Analysis. Retrieved from
  \url{http://books.google.com/books?id=LsPaAAAAMAAJ}
\end{itemize}

\end{frame}

\begin{frame}{Techniques}
\protect\hypertarget{techniques-1}{}

\begin{itemize}[<+->]
\item
  Greenacre, M. J. (2010). Correspondence analysis. Wiley
  Interdisciplinary Reviews: Computational Statistics, 2(5), 613--619.
  \url{https://doi.org/10.1002/wics.114}
\item
  Lebart, L., Morineau, A., \& Warwick, K. M. (1984). Multivariate
  descriptive statistical analysis: correspondence analysis and related
  techniques for large matrices. Wiley.
\end{itemize}

\end{frame}

\begin{frame}{Resampling}
\protect\hypertarget{resampling-1}{}

\begin{itemize}[<+->]
\item
  Berry, K. J., Johnston, J. E., \& Mielke, P. W. (2011). Permutation
  methods. Wiley Interdisciplinary Reviews: Computational Statistics, 3,
  527--542. \url{https://doi.org/10.1002/wics.177}
\item
  Strother, S. C., Anderson, J., Hansen, L. K., Kjems, U., Kustra, R.,
  Sidtis, J., \ldots{} Rottenberg, D. (2002). The Quantitative
  Evaluation of Functional Neuroimaging Experiments: The NPAIRS Data
  Analysis Framework. NeuroImage, 15(4), 747--771.
  \url{https://doi.org/10.1006/nimg.2001.1034}
\item
  Efron, B. (1979). Bootstrap Methods: Another Look at the Jackknife.
  The Annals of Statistics, 7(1), 1--26.
\end{itemize}

\end{frame}

\begin{frame}{Resampling}
\protect\hypertarget{resampling-2}{}

\begin{itemize}[<+->]
\item
  Chernick, M. R. (2008). Bootstrap methods: A guide for practitioners
  and researchers (Vol. 619). Wiley-Interscience.
\item
  Hesterberg, T. (2011). Bootstrap. Wiley Interdisciplinary Reviews:
  Computational Statistics, 3, 497--526.
  \url{https://doi.org/10.1002/wics.182}
\item
  McIntosh, A. R., \& Lobaugh, N. J. (2004). Partial least squares
  analysis of neuroimaging data: applications and advances. Neuroimage,
  23, S250--S263.
\end{itemize}

\end{frame}

\begin{frame}{Data}
\protect\hypertarget{data-1}{}

\begin{itemize}[<+->]
\item
  Escofier, B. (1978). Analyse factorielle et distances répondant au
  principe d'équivalence distributionnelle. Revue de Statistique
  Appliquée, 26(4), 29--37.
\item
  Escofier, B. (1979). Traitement simultané de variables qualitatives et
  quantitatives en analyse factorielle. Cahiers de l'Analyse Des
  Données, 4(2), 137--146.
\item
  Greenacre, M. (2014). Data Doubling and Fuzzy Coding. In J. Blasius \&
  M. Greenacre (Eds.), Visualization and Verbalization of Data
  (pp.~239--253). Philadelphia, PA, USA: CRC Press.
\end{itemize}

\end{frame}

\end{document}

\PassOptionsToPackage{unicode=true}{hyperref} % options for packages loaded elsewhere
\PassOptionsToPackage{hyphens}{url}
%
\documentclass[
  ignorenonframetext,
]{beamer}
\usepackage{pgfpages}
\setbeamertemplate{caption}[numbered]
\setbeamertemplate{caption label separator}{: }
\setbeamercolor{caption name}{fg=normal text.fg}
\beamertemplatenavigationsymbolsempty
% Prevent slide breaks in the middle of a paragraph:
\widowpenalties 1 10000
\raggedbottom
\setbeamertemplate{part page}{
  \centering
  \begin{beamercolorbox}[sep=16pt,center]{part title}
    \usebeamerfont{part title}\insertpart\par
  \end{beamercolorbox}
}
\setbeamertemplate{section page}{
  \centering
  \begin{beamercolorbox}[sep=12pt,center]{part title}
    \usebeamerfont{section title}\insertsection\par
  \end{beamercolorbox}
}
\setbeamertemplate{subsection page}{
  \centering
  \begin{beamercolorbox}[sep=8pt,center]{part title}
    \usebeamerfont{subsection title}\insertsubsection\par
  \end{beamercolorbox}
}
\AtBeginPart{
  \frame{\partpage}
}
\AtBeginSection{
  \ifbibliography
  \else
    \frame{\sectionpage}
  \fi
}
\AtBeginSubsection{
  \frame{\subsectionpage}
}
\usepackage{lmodern}
\usepackage{amssymb,amsmath}
\usepackage{ifxetex,ifluatex}
\ifnum 0\ifxetex 1\fi\ifluatex 1\fi=0 % if pdftex
  \usepackage[T1]{fontenc}
  \usepackage[utf8]{inputenc}
  \usepackage{textcomp} % provides euro and other symbols
\else % if luatex or xelatex
  \usepackage{unicode-math}
  \defaultfontfeatures{Scale=MatchLowercase}
  \defaultfontfeatures[\rmfamily]{Ligatures=TeX,Scale=1}
\fi
\usetheme[]{Berlin}
\usecolortheme{seagull}
\usefonttheme{structurebold}
% use upquote if available, for straight quotes in verbatim environments
\IfFileExists{upquote.sty}{\usepackage{upquote}}{}
\IfFileExists{microtype.sty}{% use microtype if available
  \usepackage[]{microtype}
  \UseMicrotypeSet[protrusion]{basicmath} % disable protrusion for tt fonts
}{}
\makeatletter
\@ifundefined{KOMAClassName}{% if non-KOMA class
  \IfFileExists{parskip.sty}{%
    \usepackage{parskip}
  }{% else
    \setlength{\parindent}{0pt}
    \setlength{\parskip}{6pt plus 2pt minus 1pt}}
}{% if KOMA class
  \KOMAoptions{parskip=half}}
\makeatother
\usepackage{xcolor}
\IfFileExists{xurl.sty}{\usepackage{xurl}}{} % add URL line breaks if available
\IfFileExists{bookmark.sty}{\usepackage{bookmark}}{\usepackage{hyperref}}
\hypersetup{
  pdftitle={Principal Components and Multiple Correspondence Analyses},
  pdfauthor={Derek Beaton},
  pdfborder={0 0 0},
  breaklinks=true}
\urlstyle{same}  % don't use monospace font for urls
\newif\ifbibliography
\usepackage{graphicx,grffile}
\makeatletter
\def\maxwidth{\ifdim\Gin@nat@width>\linewidth\linewidth\else\Gin@nat@width\fi}
\def\maxheight{\ifdim\Gin@nat@height>\textheight\textheight\else\Gin@nat@height\fi}
\makeatother
% Scale images if necessary, so that they will not overflow the page
% margins by default, and it is still possible to overwrite the defaults
% using explicit options in \includegraphics[width, height, ...]{}
\setkeys{Gin}{width=\maxwidth,height=\maxheight,keepaspectratio}
\setlength{\emergencystretch}{3em}  % prevent overfull lines
\providecommand{\tightlist}{%
  \setlength{\itemsep}{0pt}\setlength{\parskip}{0pt}}
\setcounter{secnumdepth}{-2}

% set default figure placement to htbp
\makeatletter
\def\fps@figure{htbp}
\makeatother

\usepackage{amssymb}
\usepackage{amsmath}

\title{Principal Components and Multiple Correspondence Analyses}
\subtitle{With resampling for stability assessment}
\author{Derek Beaton}
\date{2019APR29}
\institute{RRI RTC}

\begin{document}
\frame{\titlepage}

\hypertarget{introduction}{%
\section{Introduction}\label{introduction}}

\hypertarget{something}{%
\subsection{Something}\label{something}}

\begin{frame}{Hello}
\protect\hypertarget{hello}{}

This stuff

\end{frame}

\begin{frame}{Hell-no}
\protect\hypertarget{hell-no}{}

This stuff

\end{frame}

\hypertarget{something-else}{%
\subsection{Something else}\label{something-else}}

\begin{frame}{Hello}
\protect\hypertarget{hello-1}{}

This stuff

\end{frame}

\hypertarget{background}{%
\subsection{Background}\label{background}}

\begin{frame}{History}
\protect\hypertarget{history}{}

This stuff

\end{frame}

\begin{frame}{Not sure}
\protect\hypertarget{not-sure}{}

This stuff

\end{frame}

\hypertarget{principal-components-analysis}{%
\section{Principal Components
Analysis}\label{principal-components-analysis}}

\hypertarget{how-it-works}{%
\subsection{How it works}\label{how-it-works}}

\begin{frame}{Hello}
\protect\hypertarget{hello-2}{}

\includegraphics{test_files/figure-beamer/unnamed-chunk-1-1.pdf}
\includegraphics{test_files/figure-beamer/unnamed-chunk-1-2.pdf}

\end{frame}

\begin{frame}

\includegraphics{test_files/figure-beamer/unnamed-chunk-2-1.pdf}

\end{frame}

\begin{frame}

\includegraphics{test_files/figure-beamer/unnamed-chunk-3-1.pdf}

\end{frame}

\begin{frame}

\includegraphics{test_files/figure-beamer/unnamed-chunk-4-1.pdf}

\end{frame}

\begin{frame}

\includegraphics{test_files/figure-beamer/unnamed-chunk-5-1.pdf}

\end{frame}

\hypertarget{interpretation}{%
\subsection{Interpretation}\label{interpretation}}

\begin{frame}{Hello}
\protect\hypertarget{hello-3}{}

This stuff

\end{frame}

\hypertarget{eigen-magic}{%
\subsection{Eigen-magic}\label{eigen-magic}}

\begin{frame}{Hello}
\protect\hypertarget{hello-4}{}

This stuff

\end{frame}

\hypertarget{a-real-example}{%
\subsection{A real example}\label{a-real-example}}

\begin{frame}{Hello}
\protect\hypertarget{hello-5}{}

This stuff

\end{frame}

\begin{frame}{Hello}
\protect\hypertarget{hello-6}{}

This stuff

\end{frame}

\begin{frame}{Hello}
\protect\hypertarget{hello-7}{}

This stuff

\end{frame}

\begin{frame}{Hello}
\protect\hypertarget{hello-8}{}

This stuff

\end{frame}

\hypertarget{multiple-correspondence-analysis}{%
\section{Multiple Correspondence
Analysis}\label{multiple-correspondence-analysis}}

\hypertarget{something-1}{%
\subsection{Something}\label{something-1}}

\begin{frame}{Hello}
\protect\hypertarget{hello-9}{}

This stuff

\end{frame}

\begin{frame}{Hell-no}
\protect\hypertarget{hell-no-1}{}

This stuff

\end{frame}

\hypertarget{something-else-1}{%
\subsection{Something else}\label{something-else-1}}

\begin{frame}{Hello}
\protect\hypertarget{hello-10}{}

This stuff

\end{frame}

\hypertarget{resampling}{%
\section{Resampling}\label{resampling}}

\hypertarget{something-2}{%
\subsection{Something}\label{something-2}}

\begin{frame}{Hello}
\protect\hypertarget{hello-11}{}

This stuff

\end{frame}

\begin{frame}{Hell-no}
\protect\hypertarget{hell-no-2}{}

This stuff

\end{frame}

\hypertarget{something-else-2}{%
\subsection{Something else}\label{something-else-2}}

\begin{frame}{Hello}
\protect\hypertarget{hello-12}{}

This stuff

\end{frame}

\hypertarget{so-much-more}{%
\section{So much more}\label{so-much-more}}

\hypertarget{something-3}{%
\subsection{Something}\label{something-3}}

\begin{frame}{Hello}
\protect\hypertarget{hello-13}{}

This stuff

\end{frame}

\begin{frame}{Hell-no}
\protect\hypertarget{hell-no-3}{}

This stuff

\end{frame}

\hypertarget{something-else-3}{%
\subsection{Something else}\label{something-else-3}}

\begin{frame}{Hello}
\protect\hypertarget{hello-14}{}

This stuff

\end{frame}

\end{document}
